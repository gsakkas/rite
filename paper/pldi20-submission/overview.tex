\section{Overview}
\label{sec:overview}

We begin with an overview of our approach to suggesting fixes for faulty
programs by learning from the processes novice programmers follow
to fix errors in their programs.

\begin{figure}[ht]
\begin{ecode}
let rec mulByDigit i l =
  match l with
  | []     -> []
  | hd::tl -> (hd * i) @ mulByDigit i tl
\end{ecode}

\begin{ecode}
let rec mulByDigit i l =
  match l with
  | []     -> []
  | hd::tl -> [hd * i] @ mulByDigit i tl
\end{ecode}
\caption{(top) An ill-typed \ocaml program that should multiply each element
of a list by an integer. (bottom) The fixed version by the student.}
\label{fig:mulByDigit}
\end{figure}


\mypara{The Problem} Consider the program \mbd shown at the top of
\autoref{fig:mulByDigit}, written by a student in an undergraduate Programming
course. The program is meant to multiply all the numbers in a list with an
integer digit. The student accidentally misuses the list append operator
(\texttt{@}), applying it to a number and a list rather than two lists.
%
Novice students who are still building a mental model of how the type checker
works are often perplexed by the compiler's error message \citep{Munson_2016}.
Hence a novice will often take a long time to arrive at a suitable fix, such as
the one shown at the bottom of \autoref{fig:mulByDigit}, where \texttt{@} is
used with a singleton list containing the multiplication of the head \texttt{hd}
and \texttt{i}.
%
Our goal is to use historical data of how programmers have fixed similar errors
in their programs to automatically and rapidly guide novices to come up with
candidate solutions like the one above.


\mypara{Solution: Analytic Program Repair}
%
One approach is to view the search for candidate repairs
as a synthesis problem: synthesize a (small) set of edits
to the program that yields a good (\eg type-correct) one.
%
To ensure that synthesis is tractable, the search must be
carefully restricted to a relatively small, manually-constructed
``repair model'' that may not include the ``right'' fixes for
an erroneous program.
%FIXME: We are not manually constructing the repair model? I might make this
% sentence more clear
%
In this work, we present a novel strategy called
\emph{Analytic Program Repair} which breaks the
problem into three parts:
%
First, \emph{learn} a set of widely used \emph{fix templates}.
%
Second, \emph{predict}, for each erroneous program, the correct fix template to apply.
%
Third, \emph{synthesize} candidate repairs from the predicted template.

In the remainder of this section, we give a high-level overview
of our approach by describing how to:

\begin{enumerate}

  \item Represent fixes abstractly via \emph{fix templates}
        (\S~\ref{sec:overview:templates}),

  \item Acquire a \emph{training set} of labeled ill-typed programs and fixes
        (\S~\ref{sec:overview:data}),

  \item Learn a small set of candidate fix templates by \emph{partitioning}
        the training set (\S~\ref{sec:overview:learn}),

  \item Predict the appropriate template to apply by training a
        \emph{multi-class classifier} from the training set
        (\S~\ref{sec:overview:predict}), and

  \item Synthesize fixes by enumerating and checking terms from the
        predicted templates to give the programmer localized feedback
        (\S~\ref{sec:overview:synthesis}).
\end{enumerate}

\subsection{Representing Fixes}
\label{sec:overview:templates}

Our notion of a fix is defined as a \emph{replacement} of an existing expression
with a new \emph{candidate} expression at a specific program location. For
example, the \mbd program is fixed by replacing |(hd * i)| with the candidate
expression |[hd * i]| on line 4. We focus on AST-level replacements as they are
compact yet expressive enough to represent fixes.


\mypara{Generic Abstract Syntax Trees}
%
We represent the different possible candidate expressions via abstract fix
templates called \emph{Generic Abstract Syntax Trees} (GAST) which each
corresponds to many possible candidate expressions.
%
GASTs are obtained from concrete ASTs in two steps.
%
First, we abstract concrete variable, function, and operator names.
%
Next, we prune GASTs at a certain depth $d$ to keep only the top-level changes
of the fix. Pruned sub-trees are replaced with \emph{holes}, which can represent
\emph{any} possible expression in our language.


Together, these steps ensure that GASTs only contain information about a fix's
\emph{structure} rather than the specific changes in variables and functions.
%
For example, the fix |[hd * i]| in the \mbd example is represented by the
GAST of the expression |[_ $\oplus$ _]|, where variables |hd| and |i| are
abstracted into holes (\eg by pruning the GAST at a depth $d=2$) and |*| is
represented by an abstract binary operator. Our GAST approach is similar to
that of Lerner \emph{et al.}~\citep{Lerner2007-dt}, where AST-level modifications are used, but our proposed
GASTs represent more abstract fix schemas.


\subsection{Acquiring a Fix-Labeled Training Set}
\label{sec:overview:data}

Previous work has used experts to create a set of ill-typed programs and their
fixed versions~\citep[][]{Lerner2007-dt, Loncaric2016-uk}, or to manually create
\emph{fix templates}~\cite{kim13par} that can yield \emph{repair
patches}~\citep[][]{martinez2013automatically,martinez2015mining}.
%
These approaches are hard to scale up to yield datasets suitable for machine
learning. Also, they do not discover the \emph{frequency} in practice of particular
classes of novice mistakes and their fixes.
%
In contrast, we show that such fix templates can be \emph{learned} from a large,
automatically constructed training set of ill-typed programs labeled with their
repairs.
%
Fixes in our dataset are represented as the ASTs of the expressions that students
changed in the ill-typed program to transform it into the correct solution.

\mypara{Interaction Traces}
Following Seidel and Jhala~\citep{Seidel:2017}, we extract a labeled
dataset of erroneous programs
and their fixed versions from \emph{interaction traces}. Usually students write
several versions of their programs until they reach the correct solution for a
programming assignment. An instrumented compiler is used to capture such
sequences (or \emph{traces}) of student programs. The first type-correct
solution in this sequence of attempts is considered to be the fixed
version of all the previous ones and thus a pair for each of them is added to
the dataset. For each program pair, we then produce a \emph{diff} of their
abstract syntax trees (ASTs), and assign as the dataset's fix labels the
\emph{smallest} sub-tree that changed between the correct and ill-typed attempt
of the program.


\subsection{Learning Candidate Fix Templates}
\label{sec:overview:learn}

Each labeled program in our dataset contains a fix, which we abstract to a fix
template. For example, in the \mbd program from \autoref{fig:mulByDigit} we get
the fix candidate |[hd * i]| and hence the fix template |[_ $\oplus$ _]|.
However, a large dataset of fix-labeled programs, which may include many
diverse solutions, can introduce a huge set of fix templates, which can be
inappropriate for predicting the correct one to be used for the final program
repair.

Therefore, the next step in our approach is to learn a set of fix templates
that is \emph{small enough} to automatically predict which template to apply to
a given erroneous program, but nevertheless \emph{covers} most of the fixes that
arise in practice.

\mypara{Partitioning the Fixes} We learn a suitable small set of fix
templates by \emph{partitioning} all the templates obtained from our dataset,
and then selecting a single GAST to represent the fix templates from each fix
template set.
%
The partitioning serves two purposes.
%
First, it identifies a small set of the most common fix templates which then
enables the use of discrete classification algorithms to predict which template
to apply to a new program.
%
Second, it allows for the principled removal of outliers that arise because
student submissions often contain non-standard or idiosyncratic solutions that
we do not wish to use for suggesting fixes.

Unlike previous repair approaches that have used clustering to group together
similar programs (e.g.,~\citep{Wang_2018, Gulwani_2018}), we partition our set of
fix templates into their \emph{equivalence classes} based on a fix similarity
relation.


\subsection{Predicting Templates via Multi-classification}
\label{sec:overview:predict}

Next, we train models that can correctly predict error locations and fix
templates for a given ill-typed program. We use these models to generate
candidate expressions as possible program fixes. To reduce the
complexity of predicting the correct fix templates and error locations, we
separate these problems and encode them into two distinct \emph{supervised
classification} problems.

\mypara{Supervised Multi-Class Classification}
We propose using a \emph{supervised multi-class classification} problem for
predicting fix templates. A \emph{supervised} learning problem is one where,
given a labeled training set, the task is to learn a function that accurately
maps the inputs to output labels and generalizes to future inputs. In a
\emph{classification} problem, the function we are trying to learn maps inputs
to a discrete set of two or more output labels, called \emph{classes}.
Therefore, we encode the task of learning a function that will map
subexpressions of ill-typed programs to a small set of fix templates as a
\emph{multi-class} classification (MCC) problem.

\mypara{Feature Extraction} The machine learning models that we will train to
solve our MCC problem expect datasets of labeled \emph{fixed-length vectors} as
inputs. Therefore, we define a transformation of fix-labeled programs to
fixed-length vectors. Similarly to Seidel \emph{et al.}~\citep{Seidel:2017}, we
define a set of feature extraction functions $f_1, \ldots, f_n$, that map
program subexpressions to a numeric value (or just $\{0, 1\}$ to encode a
boolean property). Given a set of feature extraction functions, we can represent
a single program's AST as a set of fixed-length vectors by decomposing the AST
$e$ into a set of its constituent subexpressions $\{e_1, \ldots, e_m\}$ and then
representing each $e_i$ with the $n$-dimensional vector $[f_1(e_i), \ldots,
f_n(e_i)]$. This method is known as a \emph{bag-of-abstracted-terms} (BOAT)
representation in previous work \citep{Seidel:2017}.

\mypara{Predicting Templates via MCC}
Our fix-labeled dataset can be updated so the labels represent the corresponding
template that fixes each location, drawn from the minimal set of fix templates
that were acquired through partitioning. We then train a \emph{Deep Neural
Network (DNN)} classifier on the updated template-labeled data set.

Neural networks have the advantage of associating each class with a
\emph{confidence score} that can be interpreted as the model's confidence of
each class being correct for a given input. This confidence score can be used to
rank fix-template predictions for new programs and use them in descending order
to synthesize repairs until our high-level goal is reached. %FIXME: I might respecify this goal
Exploiting recent advances in machine learning, we use deep and dense
architectures \citep{Schmidhuber_2015} for more accurate fix template predictions.

\mypara{Error Localization} We view the problem of finding error locations
in a new program as a \emph{binary} classification problem. In contrast with the
template prediction problem, we want to learn a function that maps a program's
subexpressions to a binary output representing the presence of an error or not.
Therefore, this problem is a equivalent to a MCC with only two classes and thus,
we use similar deep architectures of neural networks. For each expression in a
given program, the learned model outputs a confidence score representing how
likely it is an error location that needs to be fixed. We exploit those scores
to synthesize candidate expressions for each location in descending order of
confidence.
% We use this approach, rather than standard fault localization, because FIXME.

\subsection{Synthesizing Feedback from Templates}
\label{sec:overview:synthesis}

Next, we use existing program synthesis techniques to \emph{synthesize}
candidate expressions that will be used to provide feedback to users.
Guided by our learned fix-template predictions and a set of
possible error locations to guide program synthesis, we return a ranked
list of \emph{minimal} repairs to users as feedback.

\mypara{Program Synthesis} Given a set of locations and candidate templates for
those locations, we are trying to solve a problem of \emph{program synthesis}.
For each program location, we search over all possible
expressions in the language's grammar for a small set of candidate
expressions that match the fix template and make the program type-check.
Expressions from the ill-typed program are also used during synthesis
to prune the search space of candidate expressions.

\mypara{Synthesis for Multiple Locations}
It is often the case that more than one location needs to be fixed. Therefore,
we do not only consider the ordered set of single error locations for synthesis,
but rather its power set. For simplicity, we consider fixing different program
locations as independent; the probability we assign that a set of locations
needs to be fixed is thus the product their individual confidence scores. This
is unlike recent approaches to multi-hunk program repair~\citep{Saha_2019}
where modifications depend on each other.

\mypara{Ranking Fixes} Finally, we rank each solution by two metrics, the
\emph{tree-edit distance} and the \emph{string-edit} distance. Previous work
\citep{Lerner2007-dt, Wang_2018, Gulwani_2018} has used such metrics to consider
minimal changes, \ie changes that are as close as possible to the original
programs, so novice programmers are presented with more coherent feedback.
% Additionally, more experienced programmers might have in mind what \emph{type}
% they want their functions to be. We thus provide the user with the option to
% give the intended type for the program's functions we are trying to repair. In
% our evaluation, we acquire the intended types from the fixed versions of the
% dataset (see \autoref{sec:synthesis}).

\begin{figure}[ht]
  \begin{ecode}
  let rec mulByDigit i l =
    match l with
    | []     -> []
    | hd::tl -> [(*@$v_1$@*) * (*@$v_2$@*)] @ mulByDigit i tl
  \end{ecode}
  \caption{A candidate repair for the \mbd program.}
  \label{fig:repair}
  \end{figure}

\mypara{Example} We see in \autoref{fig:repair} a minimal repair that our
method could return (|[$v_1$ * $v_2$]| in line 4), using the template discussed
in \S~\ref{sec:overview:learn} to synthesize this solution. While this solution
is not the highest-ranked that our implementation returns (which would be
identical to the human solution), it demonstrates relevant aspects of the
synthesizer. In particular, this solution has some abstracted variables, $v_1$
and $v_2$. Our algorithm suggests to the user that they can replace the two
variables with two distinct variables and insert the whole expression into a
list, in order to obtain the correct program. We hypothesize that such solutions
produced by our algorithm can provide valuable feedback to novices, and we
investigate that claim empirically in \S~\ref{sec:eval:useful}.
