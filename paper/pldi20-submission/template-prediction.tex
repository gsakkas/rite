\section{Predicting Fix Templates}
\label{sec:templ-pred}

Given a candidate set of templates, our next task is to \emph{train} a model
that, when given an (erroneous) program, can predict which template to use for
each location in that program.
%
We do so by defining a function $\predictsym$ which takes as input
%
(1)~a feature extraction function $\featuresym$,
%
(2) a dataset $\datasetsym$ of program pairs $(\pbad,\ \pfix)$,
%
(3)~a list of fix templates $\T$;
%
and returns as output a \emph{fix-template-predictor}. This predictor, given an
erroneous program, returns the locations of likely fixes, and the templates to
be applied at those locations.

We build $\predictsym$ using three helper functions that carry out each of the
high-level steps.
%
First, the $\extractsym$ function extracts \emph{features} and \emph{labels}
from the program pair dataset. Next, these feature vectors are grouped and fed
into $\trainsym$ which produces two models, $\Model$ and $\ModelT$, that are
respectively used for error localization and predicting fix templates.
%
Finally, $\ranksym$ takes the features for a new (erroneous) program and
queries the trained models to return the likely fix locations and corresponding
fix-templates.

Next, we describe the key data-types in \autoref{fig:api}, our implementations
of the three key steps, and how they are combined to yield the $\predictsym$
algorithm.

\paragraph{Confidences, Data and Labels.}
As shown in \autoref{fig:api}, we define $\Emap{a}$ as a mapping from
expressions $e$ to values of type $a$, and $\Tmap{a}$ as a mapping from
templates $\T$ to such values. For example, $\Tmap{\Conf}$ is a mapping from
templates $\T$ to their confidence scores $\Conf$. $\datasym$ represents feature
vectors used to train our predictive models, while $\labelsym{\ConfBin}$ are the
dataset labels for training and $\labelsym{\Conf}$ are the output confidence
scores. Finally, $\pairsym$ is a program pair $(\pbad,\ \pfix)$.

\paragraph{Features and Predictors.}
We define $\featuresym$ as a function that generates the feature vectors
$\datasym$ for each subexpression of an input program $e$. Those feature vectors
are given in the form of a map $\Emap{\datasym}$, which maps all subexpressions
of the input program $e$ to its feature vector $\datasym$.

$\predictorsym$s are learned fix-template-predictors returned from our algorithm
that are used to generate confidence score mappings for input programs $e$.
Specifically, they return a map $\Emap{(\labelsym{\Conf})}$ that associates each
subexpression of the input program $e$ with a confidence score
$\labelsym{\Conf}$.
% \BC{here it's ``a confidence score $\labelsym{\Conf}$''; just recently it was
% ``their confidence scores $\Conf$''}

\paragraph{Architecture.}
First, the $\extractsym$ function takes as input the feature extraction
functions $\featuresym$, a list of templates $\List{\T}$ and a single program
pair $\pairsym$ and generates a map $\Emap{(\datasym \times
\labelsym{\ConfBin})}$ of feature vectors and boolean labels for all
subexpressions of the erroneous input program from $\pairsym$.
%
All feature vectors $\datasym$ and labels $\labelsym{\ConfBin}$ are then
accumulated into one list, which is given as input to $\trainsym$ and are used
for training the two models $\Model$ and $\ModelT$ that are respectively used
for predicting error locations and fix templates.
%
Next, the two trained models $\Model$ and $\ModelT$, along with $\datasym$ from
a new and previously unseen program, can be fed into $\ranksym$. This
produces a $\predictorsym$, which can be used to map subexpressions of the new
program to possible error locations and fix templates.

\begin{figure}
\small
\begin{minipage}[c]{\linewidth}
  \lstDeleteShortInline{|}
  \[
  \boxed{
  \begin{array}{lcl}
    \Conf         & \defeq & \{r \in \R\ |\ 0 \le r \le 1\} \\
    \ConfBin      & \defeq & \{b \in \R\ |\ b = 0 \vee b = 1\} \\
    \T            & \defeq & e^{RTL} \\
    \Emap{a}      & \defeq & e \to a \\
    \Tmap{a}      & \defeq & \T \to a \\
    \datasym      & \defeq & \List{\Conf} \\
    \labelsym{a}  & \defeq & a \times \Tmap{a} \\
    \pairsym      & \defeq & e \times e \\
    \datasetsym   & \defeq & \List{\pairsym} \\
    \midrule
    \featuresym   & \defeq & e \to \Emap{\datasym} \\
    \predictorsym & \defeq & e \to \Emap{(\labelsym{\Conf})} \\
    \midrule
    \abstrsym   & : & e \to \T \\
    \diffsym    & : & \pairsym \to \List{e} \\
    \midrule
    \extractsym & : & \featuresym \to \List{\T} \to \pairsym \\
                &   & \quad \to \Emap{(\datasym \times \labelsym{\ConfBin})} \\
    \trainsym   & : & \List{\datasym \times \labelsym{\ConfBin}} \to \Model \times \ModelT \\
    \predictsym & : & \Model \to \ModelT \to \datasym \to \labelsym{\Conf} \\
    \midrule
    \repairsym  & : & \featuresym \to \List{\T} \to \datasetsym \to \predictorsym \\
  \end{array}
  }
  \]
  \lstMakeShortInline[mathescape=true]{|}
  \captionof{figure}{
    A high-level API for converting program pairs to
    feature vectors and template labels.
  }
  \label{fig:api}
\end{minipage}
\end{figure}



\subsection{Feature and Label Extraction}
\label{sec:templ-pred:extract}
The machine learning algorithms that we use for predicting fix templates and
error locations expect fixed-length \emph{feature vectors} $\datasym$ as their
input. However, we want to repair variable-sized programs over \lang. We thus
use the $\extractsym$ function to convert programs to feature vectors.

Following \citep{Seidel:2017}, we choose to model a program as a \emph{set} of
feature vectors, where each element corresponds to a subexpression in the
program. Thus, given an erroneous program $\pbad$ we first split it into its
constituent subexpressions and then transform each subexpression into a single
feature vector, \ie $\featuresym\ \pbad :: \Emap{\datasym}$. We show here
the five major feature categories used.

\paragraph{Local syntactic features.}
These features describe the syntactic category of each expression $e$. In other
words, for each production rule of $e$ in \autoref{fig:ml-syntax} we introduce a
feature that is enabled (set to $1$) if the expression was built with that
production, and disabled (set to $0$) otherwise.

\paragraph{Contextual syntactic features.}
The \emph{context} in which an expression occurs can be critical for correctly
predicting error sources and fix templates. Therefore, we include contextual
features, which are similar to the local syntactic features but describe the
parent and children of an expression. For example, the \IsNilC feature
would describe whether an expression's \emph{first child} is []. This
is similar
to the \emph{n-grams} used in linguistic models
\citep{Hindle2012-hf,Gabel2010-el}.

\paragraph{Expression size.}
We also include a feature representing the \emph{size} of each expression, \ie
how many subexpressions does it contain? This allows the model to learn that,
\eg, expressions closer to the leaves are more likely to be fixed than
expressions closer to the root.

\paragraph{Typing features.}
The programs we are trying to repair are \emph{untypeable}, but a \emph{partial
typing} derivation from the type checker could still provide useful information
to the model. Therefore, we include \emph{typing} features in our
representation. Due to the parametric type constructors
$\tfun{\cdot}{\cdot}$, $\tprod{\cdot}{\cdot}$, and $\tlist{\cdot}$, there is an
\emph{infinite} set of possible types --- but we must have a \emph{finite} set
of features. We add features for each abstract type constructor that describes whether a
given type uses that constructor. For example, the type
$\tfun{\tint}{\tfun{\tint}{\tbool}}$
would enable the $\tfun{\cdot}{\cdot}$, $\tint$, and $\tbool$ features.

We add these features for parent and child expressions to summarize the context,
but also for the current expression, as the type of an expression is not always
clear syntactically.

% TODO: cite the correct paper
\paragraph{Type error slice.}
We wish to distinguish changes that could fix the error
from changes that \emph{cannot possibly} fix the error. Thus, we compute a
minimal type-error \emph{slice} (\eg \citep{Tip2001-qp,Haack2003-vc}) for the
program (\ie the set of expressions that contribute to the error) and if the
program contains multiple type-errors, we compute a minimal slice for each
error. We then have a post-processing step that discards all expressions that
are not included in those slices.

\paragraph{Labels.}
Recall that we use two predictive models, $\Model$ for error localization and
$\ModelT$ for predicting fix templates. We thus require two sets of
\emph{labels} associated with each feature vector that are given by
$\labelsym{\ConfBin}$. $\Model$ is trained using the set $\List{\datasym \times
\ConfBin}$, while $\ModelT$ using the set $\List{\datasym \times
\Tmap{\ConfBin}}$.

$\Model$'s labels of type $\ConfBin$ are set to ``true'' for each subexpression
of a program $\pbad$ that changed in $\pfix$. A label $\Tmap{\ConfBin}$, for a
subexpression of $\pbad$, maps to the repair template $\T$ that was used to fix
it. $\Tmap{\ConfBin}$ associates all subexpressions with a fixed number of
templates $\List{\T}$ given as input to $\extractsym$. Therefore, for the
purpose of template prediction, $\Tmap{\ConfBin}$ can be viewed as a
fixed-length boolean vector that represents the fix templates used to repair
each subexpression. This vector has at most one slot set to ``true'',
representing the template used to fix $\pbad$. These labels are extracted using
$\diffsym$ and $\abstrsym$, similarly to the way that templates were extracted
in \autoref{sec:templ-partition:templates}.


\subsection{Training Predictive Models}
\label{sec:templ-pred:train}
\lstDeleteShortInline{|} % sigh...

Our goal with the $\trainsym$ function is to train two separate
\emph{classifiers} given a training set $\List{\datasym \times
\labelsym{\ConfBin}}$ of labeled examples. $\Model$ predicts error locations and
$\ModelT$ predicts fix templates for a new input program $\pbad$. Critically, we
require that the error localization classifier outputs a \emph{confidence score}
$\Conf$ that represents the probability that the subexpression is the error that
needs to be fixed. We also require that the fix template classifier outputs a
confidence score $\Conf$ for each of the selected fix templates that measures
how sure the classifier is that a given template can be used to repair the
associated location of the input program $\pbad$.

We consider a standard learning algorithm to generate our models: \emph{neural
networks}. A thorough introduction to neural networks can be found in
\citep{Hastie2009-bn,Nielsen2015-pu}.

\paragraph{Neural Networks}
The model that we use is a type of neural network called a \emph{multi-layer
perceptron}. A multi-layer perceptron can be represented as a directed acyclic
graph whose nodes are arranged in layers that are fully connected by weighted
edges. The first layer corresponds to the input features, and the final to the
output. The output of an internal node is given as the sum of the weighted
outputs of the previous layer passed to a non-linear function, called the
activation function. The number of layers,
the number of nodes per layer, and the connections between layers constitute the
\emph{architecture} of a neural network. In this work, we use relatively
\emph{deep neural networks} (\dnn). Finally, we can train a \dnn $\Model$ as a
binary classifier, which will predict whether a location in a program $\pbad$
has to be fixed or not.

\paragraph{Multi-class \dnn{}s}
While the above model is enough for error localization, in the case of template
prediction we have to select from more than two \emph{classes}. We again use a
\dnn for our template prediction $\ModelT$, but we adjust the output layer to
have $N$ nodes for the $N$ chosen template-classes. For multi-class
classification problems solved with neural networks, usually a \emph{softmax}
function is used at output layer \citep{Goodfellow-et-al-2016,Bishop-book-2006}.
Softmax assigns probabilities to each class that must add up to 1.
This additional constraint speeds up training.

% For an output vector $y = (y_1, \dots, y_N) \in \R^{N}$, the standard softmax
% function is defined as:
% \[ \sigma(y)_i = \frac{e^{y_i}}{\sum_{j=1}^{N} e^{y_j}},\ for\ i = 1, \dots, N \]


\subsection{Predicting Fix Templates}
\label{sec:templ-pred:predict}

Our ultimate goal is to be able to pinpoint what parts of an erroneous program
should be repaired and what fix templates should be utilized for that purpose.
Therefore, the $\predictsym$ function uses $\ranksym$ to predict all
subexpressions' confidence scores $\Conf$ to be an error location and confidence
scores $\Tmap{\Conf}$ for each of the selected fix templates. In this section we
show how all the functions in our high-level API in \autoref{fig:api} are
combined to produce a final list of confidence scores for a new program $p$.
\autoref{algo:predict-algo} shows our high-level $\predictsym$ algorithm.

\begin{algorithm}[t]
    \caption{Predicting Templates Algorithm}
    \label{algo:predict-algo}
    \renewcommand{\algorithmicrequire}{\textbf{Input:}}
    \renewcommand{\algorithmicensure}{\textbf{Output:}}
    \begin{algorithmic}[1]
    \Require{Feature Extraction Functions $F$, Fix Templates $Ts$, Program Pair Dataset $D$}
    \Ensure{Predictor $Pr$}
    \Procedure{Predict}{$F,\,Ts,\,D$}
    \State $D_{ML} \gets \emptyset$
    \ForAll{$\pbad \times \pfix \in D$}
      \State $d \gets$ \Call{Extract}{$F,\,Ts,\,\pbad \times \pfix$}
      \State $D_{ML} \gets D_{ML}\,\cup$ \Call{InSlice}{$\pbad,\,d$}
    \EndFor
    \State $Models \gets$ \Call{Train}{$D_{ML}$}
    \State $Data \gets \lambda p.$ \Call{InSlice}{$p,$ \Call{Extract}{$F,\,Ts,\,p \times p$}}
    \State $Pr \gets \lambda p.$ \Call{Map}{$\lambda \tilde{p}.$ \Call{Rank}{$Models,\,\tilde{p}[0]$}, $Data(p)$}
    \State \Return{$Pr$}
    \EndProcedure
    \end{algorithmic}
\end{algorithm}


\paragraph{The Prediction Algorithm}
Our algorithm first extracts the machine-learning-amenable dataset $D_{ML}$
from the program pairs dataset $D$. For each program pair in $D$,
\textsc{Extract} returns a mapping from the erroneous program's subexpressions
to features and labels. Then, \textsc{InSlice} keeps only the expressions in
the the type-error slice and evaluates to a list of the respective feature and
label vectors, which is added to the $D_{ML}$ dataset. This dataset
is used by the \textsc{Train} function to generate our predictive $Models$, \ie
$\Model$ and $\ModelT$.

At this point we want to generate a $\predictorsym$ for a new unknown program
$p$. We perform feature extraction for $p$ with \textsc{Extract}, and use
\textsc{InSlice} to restrict to expressions in $p$'s type-error slice.
The result is given by $Data(p)$.

\textsc{Rank} is then applied to all subexpressions produced by $Data(p)$ with
$\textsc{Map}$, which will create a mapping of the type
$\Emap{(\labelsym{\Conf})}$ associating expressions with confidence scores. We
apply \textsc{Rank} to each feature vector that corresponds to an expression in
the type-error slice of $p$. These vectors are the first elements of $\tilde{p}
\in Data(p)$, which are of type $\datasym \times \labelsym{\ConfBin}$. Finally,
$\predictorsym\ Pr$ is returned, which is used by our synthesis algorithm in
\autoref{sec:synthesis} to correlate subexpressions in $p$ with their confidence
scores.
