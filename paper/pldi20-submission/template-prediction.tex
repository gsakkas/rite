\section{Predicting Fix Templates}
\label{sec:templ-pred}

Given a candidate set of templates, our next task is to \emph{train} a model
that, when given an (erroneous) program, can predict which template to use for
each location in that program.
%
We do so by defining the $\repairsym$ function which takes as input
%
(1)~a feature extraction function $\featuresym$,
%
(2) a dataset $\datasetsym$ of program pairs $(\pbad,\ \pfix)$,
%
(3)~a list of fix templates $\T$,
%
and returns as output a \emph{fix-template-predictor} that, given an erroneous
program, returns the locations of likely fixes, and the templates to be applied
at those locations.

We build $\repairsym$ using three helper functions that carry out each of the
high-level steps.
%
First, the $\extractsym$ function extracts \emph{features} and \emph{labels}
from the program pair dataset. Next, these feature vectors are grouped and fed
into $\trainsym$ which produces two models $\Model$ and $\ModelT$ that are
respectively used for error localization and predicting fix templates.
%
Finally, $\predictsym$ takes the features for a new (erroneous) program and
queries the trained models to return the likely fix locations and corresponding
fix-templates.

Next, we describe the key data-types, our implementations of the three key
steps, and how they are combined to yield the $\repairsym$ algorithm.

\paragraph{Confidences, Data and Labels}
We define $\Conf$ as real numbers between $0$ and $1$, which are used for
(normalized) input feature vectors and output confidence scores. $\ConfBin$ are
their boolean counterparts and are used for labelling the dataset. $\Emap{a}$
represents a mapping from expressions $e$ to values of type $a$, while
$\Tmap{a}$ represents a mapping from templates $\T$ to such values. For example,
$\Tmap{\Conf}$ is a mapping from templates $\T$ to their confidence scores
$\Conf$.

Additionally, $\datasym$ represents feature vectors used to train our predict
models, while $\labelsym{\ConfBin}$ are the dataset labels for training and
$\labelsym{\Conf}$ are the output confidence scores. Finally, $\pairsym$ are
program pairs of the nature $(\pbad,\ \pfix)$, containing the erroneous program
$\pbad$ and its fix $\pfix$.

\paragraph{Features and Predictors}
We define here $\featuresym$ as a function that generates the feature vectors
$\datasym$ for each subexpression from a input program $e$. Those feature
vectors are given in the form of a map $\Emap{\datasym}$, which maps all
subexpressions of the input program $e$ to its feature vector $\datasym$.

$\predictorsym$s are learned fix-template-predictors returned from our algorithm
that are used to generate confidence scores mappings for input programs $e$.
Specifically, they return a map $\Emap{(\labelsym{\Conf})}$ that associates each
subexpression of the input program $e$ with a confidence score
$\labelsym{\Conf}$.

\paragraph{Architecture}
First, the $\extractsym$ function takes as input the feature extraction
functions $\featuresym$, a list of templates $\List{\T}$ and a single program
pair $\pairsym$ and generates a map $\Emap{(\datasym \times
\labelsym{\ConfBin})}$ of feature vectors and boolean labels for all
subexpressions of the erroneous input program from $\pairsym$.
%
All feature vectors $\datasym$ and labels $\labelsym{\ConfBin}$ are then
accumulated into one list, which is given as input to $\trainsym$ and are used
as a training dataset for two separate models $\Model$ and $\ModelT$ that are
respectively used for predicting error locations and fix templates.
%
Next, the two trained models $\Model$ and $\ModelT$, along with feature vectors
$\datasym$ from a new and previously unseen program, can be fed into
$\predictsym$ in order to acquire a $\predictorsym$, which can be used to map
subexpression of the new program to possible error locations and fix templates.

\begin{figure}
\small
\begin{minipage}[c]{\linewidth}
  \lstDeleteShortInline{|}
  \[
  \boxed{
  \begin{array}{lcl}
    \Conf         & \defeq & \{r \in \R\ |\ 0 \le r \le 1\} \\
    \ConfBin      & \defeq & \{b \in \R\ |\ b = 0 \vee b = 1\} \\
    \T            & \defeq & e^{RTL} \\
    \Emap{a}      & \defeq & e \to a \\
    \Tmap{a}      & \defeq & \T \to a \\
    \datasym      & \defeq & \List{\Conf} \\
    \labelsym{a}  & \defeq & a \times \Tmap{a} \\
    \pairsym      & \defeq & e \times e \\
    \datasetsym   & \defeq & \List{\pairsym} \\
    \midrule
    \featuresym   & \defeq & e \to \Emap{\datasym} \\
    \predictorsym & \defeq & e \to \Emap{(\labelsym{\Conf})} \\
    \midrule
    \abstrsym   & : & e \to \T \\
    \diffsym    & : & \pairsym \to \List{e} \\
    \midrule
    \extractsym & : & \featuresym \to \List{\T} \to \pairsym \\
                &   & \quad \to \Emap{(\datasym \times \labelsym{\ConfBin})} \\
    \trainsym   & : & \List{\datasym \times \labelsym{\ConfBin}} \to \Model \times \ModelT \\
    \predictsym & : & \Model \to \ModelT \to \datasym \to \labelsym{\Conf} \\
    \midrule
    \repairsym  & : & \featuresym \to \List{\T} \to \datasetsym \to \predictorsym \\
  \end{array}
  }
  \]
  \lstMakeShortInline[mathescape=true]{|}
  \captionof{figure}{
    A high-level API for converting program pairs to
    feature vectors and template labels.
  }
  \label{fig:api}
\end{minipage}
\end{figure}


\begin{comment}
to be used
and maps each subexpression of the original erroneous
program to a boolean vector. The first entry refers to the error locations that
the user fixed, \ie whether a subexpression changed in a program pair. Those
locations are acquired through the $\diffsym$ function. The rest of the boolean
vector has at most one $\etrue$ value, and corresponds to the fix template $\T$
that was used to repair that subexpression.

The $\extractsym$ function is applied to all the dataset program pairs, and all
subexpression feature and label vectors are then grouped together into one
larger \emph{training} dataset. The training dataset is used by the $\trainsym$
function to produce two predictive models, $\Model$ and $\ModelT$. $\Model$ is
used for error localization and $\ModelT$ for predicting fix templates.

Finally, the $\predictsym$ function produces an error location confidence score
$\Conf$ and confidence scores $\Tmap{\Conf}$ for each of the selected fix
templates for a specific subexpression in a new program $e$ that we want to
repair. The $\predictsym$ function expects as input the two generated
predictive models and a new feature vector $\V$, that corresponds to a
subexpression's feature vector acquired again by $\extractsym$. It is then
applied to each subexpression in $e$'s type-error slice (TODO: ref) and
generates a mapping from program expressions to confidence scores $\Emap{(\Conf
\times \Tmap{\Conf})}$ as output of the top-level function $\repairsym$.
\end{comment}


\subsection{Feature and Label Extraction}
\label{subsec:extract}
The machine learning algorithms that we use for predicting fix templates and
error locations, expect fixed-length \emph{feature vectors} $\datasym$ as their
input. However, we want to repair variable-sized programs over \lang. Thus, our
first task is to convert programs to feature vectors with the $\extractsym$
function.

Following \citep{Seidel:2017}, we choose to model a program as a \emph{set} of
feature vectors, where each element corresponds to a subexpression in the
program. Thus, given an erroneous program $\pbad$ we would first split it into
its constituent subexpressions and then transform each subexpression into a
single feature vector, \ie $\featuresym\ \pbad \mapsto \Emap{\datasym}$. We show
here the five major feature categories used.

\paragraph{Local syntactic features}
These features describe the syntactic category of each expression $e$. In other
words, for each production rule of $e$ in \autoref{fig:ml-syntax} we introduce a
feature that is enabled (set to $1$) if the expression was built with that
production, and disabled (set to $0$) otherwise. For example, a \IsNil feature
would describe whether an expression is the empty list $\enil$.

% We distinguish between matching on a list vs.\ on a pair, as this affects the
% typing derivation. We also assume that all pattern matches are well-formed ---
% \ie all patterns must match on the same type. Ill-formed match expressions would
% lead to a type-error; however, they are already effectively localized to the
% match expression itself. We note that this is not a \emph{fundamental}
% limitation, and one could easily add features that specify whether a match
% \emph{contains} a particular pattern, and thus have a match expression that
% enables multiple features.

\paragraph{Contextual syntactic features}
The \emph{context} in which an expression occurs can be critical for predicting
correctly fix templates and determining if they are the source of an error.
Therefore, we include these features, which are similar to the local syntactic
features but they describe the parent and children of an expression. For
example, the \IsCaseListP feature would describe whether an expression's
\emph{parent} matches on a list. If a particular $e$ does not have children (\eg
a variable $x$) or a parent (\ie the root expression), we leave the
corresponding features disabled. This gives us a notion of the \emph{context} in
which an expression occurs, similar to the \emph{n-grams} used in linguistic
models \citep{Hindle2012-hf,Gabel2010-el}.

\paragraph{Expression size}
We also include a feature representing the \emph{size} of each expression, \ie
how many sub-expressions does it contain?
% For example, the \ExprSize feature in \autoref{fig:mulByDigit} is set to four
% for the expression |mulByDigit i tl| as it contains four expressions: the three
% variables and the application itself.
This allows the model to learn that, \eg, expressions closer to the leaves are
more likely to be fixed than expressions closer to the root.

\paragraph{Typing features}
The programs we are trying to repair are \emph{untypeable}, but a \emph{partial
typing} derivation from the type checker could still provide useful information
to the model. Therefore, we include \emph{typing} features in our
representation.

A difficulty that arises here is that, due to the parametric type constructors
$\tfun{\cdot}{\cdot}$, $\tprod{\cdot}{\cdot}$, and $\tlist{\cdot}$, there is an
\emph{infinite} set of possible types --- but we must have a \emph{finite} set
of features. Thus, we abstract the type of an expression to the set of type
constructors it \emph{mentions}, and add features for each type constructor that
describe whether a given type mentions the type constructor. For example, the
type $\tint$ would only enable the $\tint$ feature, while the type
$\tfun{\tint}{\tbool}$ would enable the $\tfun{\cdot}{\cdot}$, $\tint$, and
$\tbool$ features.

We add these features for parent and child expressions to summarize the context,
but also for the current expression, as the type of an expression is not always
clear \emph{syntactically}. For example, the expressions |tl| and
|mulByDigit i tl| in \autoref{fig:mulByDigit} both enable the \emph{list} type
feature, as they are both inferred to have a type that mentions $\tlist{\cdot}$.

% Note that our use of typing features in an ill-typed program subjects us to
% \emph{traversal bias} \citep{McAdam1998-ub}. For example, the |mulByDigit i tl|
% expression might alternatively be assigned the type $\tint$. Our models will
% have to learn good localizations in spite this bias.

\paragraph{Type error slice}
Furthermore, we wish to distinguish between changes that could fix the error,
and changes that \emph{cannot possibly} fix the error. Thus, we compute a
minimal type-error \emph{slice} for the program (\ie the set of expressions that
contribute to the error) and if the program contains multiple type-errors, we
compute a minimal slice for each error. We then have a post-processing step that
discards all expressions that are not included in those slices.


\paragraph{Labels}
Recall that we use two predictive models, $\Model$ for error localization and
$\ModelT$ for predicting fix templates, thus we require two sets of
\emph{labels} associated with each feature vector that are given by
$\labelsym{\ConfBin}$. $\Model$ is trained using the set $\List{\datasym \times
\ConfBin}$, while $\ModelT$ using the set $\List{\datasym \times
\Tmap{\ConfBin}}$. A ``true'' boolean label used for $\Model$ means the
associated expression \emph{should} change, where ``false'' means that it
\emph{should not}. We also want to assign the correct labels for $\ModelT$. A
label $\Tmap{\ConfBin}$ must map a subexpression of $\pbad$ to the repair
template $\T$ that was used to fix it. If a subexpression was not changed, then
it maps to $null$. $\Tmap{\ConfBin}$ associates all subexpressions with a fixed
number of templates $\List{\T}$ given as input to $\extractsym$. Therefore, for
the purpose of template prediction, $\Tmap{\ConfBin}$ can be a fixed-length
boolean vector that represents the fix templates used to repair each
subexpression. This vector has at most one slot set to ``true'', representing
the template that used to fix $\pbad$. These labels are extracted using
$\diffsym$ and $\abstrsym$, similarly to the way that templates were extracted
in \autoref{sec:templ-cluster:templates}.

\subsection{Training Predictive Models}
\label{subsec:train}
\lstDeleteShortInline{|} % sigh...

Our goal with the $\trainsym$ function is to train two separate
\emph{classifiers} given a training set $\List{\V \times \labelsym{\ConfBin}}$
of labeled examples, one used to predict error locations $\Model$ and the other
to predict fix templates $\ModelT$ for a new input program $\pbad$. Critically,
we require that the error localization classifier outputs a \emph{confidence
score} $\Conf$ that represents the probability that the subexpression in program
$p$ it was applied to is the error that needs to be fixed. We also require that
the fix template classifier outputs a confidence score $\Conf$ for each of the
selected fix templates that measures how sure the classifier is that a given
template can be used to repair the associated location of the input program
$\pbad$.

We consider a standard learning algorithm to generate our models: \emph{neural
networks}. A thorough introduction to neural networks can be found in
introductory machine learning textbooks \citep[\eg][]{Hastie2009-bn}.

% Below we briefly introduce each technique by describing the rules it learns, and
% summarize its advantages and disadvantages. For our application, we are
% particularly interested in three properties -- expressiveness, interpretability
% and ease of generalization. Expressiveness measures how complex prediction rules
% are allowed to be, and interpretability measures how easy it is to explain the
% cause of prediction to a human. Finally ease of generalization measures how
% easily the rule generalizes to examples that are not in the training set; a rule
% that is not easily generalizable might perform poorly on an unseen test set even
% when its training performance is high.


\paragraph{Neural Networks}
The model that we use is a type of neural network called a \emph{multi-layer
perceptron} (see \citep{Nielsen2015-pu} for an introduction to neural networks).
A multi-layer perceptron can be represented as a directed acyclic graph whose
nodes are arranged in layers that are fully connected by weighted edges. The
first layer corresponds to the input features, and the final to the output. The
output of an internal node $v$ is
\[ h_v = g\,(\sum_{j \in N(v)}\!W_{jv} h_j ) \] where $N(v)$ is the set of nodes
in the previous layer that are adjacent to $v$, $W_{jv}$ is the weight of the
$(j, v)$ edge and $h_j$ is the output of node $j$ in the previous layer. Finally
$g$ is a non-linear function, called the activation function, which in recent
work is commonly chosen to be the \emph{rectified linear unit} (ReLU), defined
as $g(x) = \mathsf{max}(0,x)$ \citep{Nair2010-xg}. The number of layers, the
number of neurons per layer, and the connections between layers constitute the
\emph{architecture} of a neural network. In this work, we use relatively
\emph{deep neural networks} (\dnn) which have an input layer, three hidden
layers and an output layer.

% A major advantage of neural networks is their ability to discover interesting
% combinations of features through non-linearity, which significantly reduces the
% need for manual feature engineering, and allows high expressivity. On the other
% hand, this makes the networks particularly difficult to interpret and also
% difficult to generalize unless vast amounts of training data are available.

\paragraph{Multi-class \dnn{}s}
While the above model is enough for error localization (since we have to select
from a binary option), some adjustments on the output layer must be applied. We
again use a deep neural networks for our template prediction $\ModelT$s, but the
output here is a $N$-length vector and therefore the output layer has $N$ nodes.
For multi-class classification problems solved with neural network techniques,
usually a \emph{softmax} function is used for the output layer
\citep{Goodfellow-et-al-2016, Bishop-book-2006}. Softmax assigns decimal
probabilities to each class that must add up to 1.0. This additional constraint
helps training converge more quickly than it otherwise would.

For an output vector $y = (y_1, \dots, y_N) \in \R^{N}$, the standard softmax
function is defined as:
\[ \sigma(y)_i = \frac{e^{y_i}}{\sum_{j=1}^{N} e^{y_j}},\ for\ i = 1, \dots, N \]

The number of neurons per layer are left as the model's hyper-parameters, but we
choose again a deep architecture with three layers, and with each layer having
more neurons than the respective error-localization models, with the exact
number needing some tuning.

% TODO: it feels like something is missing (beside the examples)


\subsection{Predicting Fix Templates}
\label{subsec:predict}

Our ultimate goal is to be able to pinpoint what parts of an erroneous program
should be repaired and what fix templates should be utilized for that purpose.
Therefore, the $\repairsym$ function uses $\predictsym$ to predict all
subexpressions' confidence scores $\Conf$ to be an error location and confidence
scores $\Tmap{\Conf}$ for each of the selected fix templates. In this section we
show how all the functions in our high-level API in \autoref{fig:api} are
combined to produce a final list of confidence scores for a new program $p$.
\autoref{algo:predict-algo} shows a high-level algorithm for $\repairsym$.

\paragraph{The Prediction Algorithm}
Our algorithm firstly extracts the machine-learning-amenable dataset $D_{ML}$
from the program pairs dataset $D$. For each program pair in $D$, $\extractsym$
returns a mapping from the erroneous program's subexpressions to features and
labels. Then, $\textsc{SubesInSlice}$ keeps only the expressions in the the
type-error slice and returns a list of the respective feature and label vectors,
which is then added to the $D_{ML}$ dataset. The $D_{ML}$ dataset is used by the
$\trainsym$ function to generate our predictive $Models$, \ie $\Model$ and
$\ModelT$.

At this point we have to perform again the feature extraction for a new
(unknown) program $p$, not included before in our dataset $D$. Thus we use again
$\extractsym$, by providing program $p$ twice, since in this case we don't have
the fixed program, as we are trying to produce it. We also acquire only the
expressions in $p$'s type-error slice with $\textsc{SubesInSlice}$, which is
given by $Data(p)$.

We then apply $\predictsym$ to all subexpressions produced by $Data(p)$ with
$\textsc{Map}$, a function that will generate a map of type
$\Emap{(\labelsym{\Conf})}$ associating expressions with confidence scores.
$\predictsym$ is applied to each feature vector that corresponds to an
expression in the type-error slice of $p$. These vectors are the first element
of $\tilde{p} \in Data(p)$, which are of type $\datasym \times
\labelsym{\ConfBin}$. Finally, the $\predictorsym\ Pr$ is returned, which is used
by our synthesis algorithm in \autoref{sec:synthesis} to correlate
subexpressions in $p$ with their confidence scores.

\begin{algorithm}[t]
    \caption{Predicting Templates Algorithm}
    \label{algo:predict-algo}
    \renewcommand{\algorithmicrequire}{\textbf{Input:}}
    \renewcommand{\algorithmicensure}{\textbf{Output:}}
    \begin{algorithmic}[1]
    \Require{Feature Extraction Functions $F$, Fix Templates $Ts$, Program Pair Dataset $D$}
    \Ensure{Predictor $Pr$}
    \Procedure{Predict}{$F,\,Ts,\,D$}
    \State $D_{ML} \gets \emptyset$
    \ForAll{$\pbad \times \pfix \in D$}
      \State $d \gets$ \Call{Extract}{$F,\,Ts,\,\pbad \times \pfix$}
      \State $D_{ML} \gets D_{ML}\,\cup$ \Call{InSlice}{$\pbad,\,d$}
    \EndFor
    \State $Models \gets$ \Call{Train}{$D_{ML}$}
    \State $Data \gets \lambda p.$ \Call{InSlice}{$p,$ \Call{Extract}{$F,\,Ts,\,p \times p$}}
    \State $Pr \gets \lambda p.$ \Call{Map}{$\lambda \tilde{p}.$ \Call{Rank}{$Models,\,\tilde{p}[0]$}, $Data(p)$}
    \State \Return{$Pr$}
    \EndProcedure
    \end{algorithmic}
\end{algorithm}

