\section{Introduction}
\label{sec:intro}

%%% Motivation and the problem
The increasing number of computing-related jobs~\citep[][]{compsci-demand}
in recent years has resulted in an unprecedented
demand for computer science education.  Computing enrollment at colleges
and universities is high~\citep[][]{compsci-classes} and is only projected
to increase in the foreseeable future. Students are augmenting traditional
education with Massive Open Online Courses (MOOCs)~\citep[][]{moocs}.
While these larger classrooms have lead to computer science education
becoming more accessible, its \emph{quality} has been
questioned~\cite{FIXME}, especially with regard to the feedback that
students receive~\cite{FIXME}.

%%% Good properties of a solution
Recent research has focused on providing \emph{fully automated feedback} to
students for classroom programming assignments (e.g.,~\cite{FIXME,FIXME}). In
this paper we propose a method to provide personalized
feedback for introductory programming exercises without requiring
additional instructor effort. We build upon insights from advances in
\emph{program repair} research to generate such feedback
(e.g.,~\cite{FIXME,FIXME}).
% TODO: Cite clara, autoproof, sarfgen, ... here

By repairing an incorrect student attempt on a programming assignment and
producing a \emph{minimal} repair \wrt some edit-distance metric,
we can generate \emph{high-quality}
feedback for that attempt. As a result, any repair algorithm we propose must be
able to \emph{generalize} over different programming assignments.
In this paper we focus on generating
repairs to enable better feedback generation.
\WRW{FIXME: WRW cannot tell what "the latter part" refers to.
and leave the latter part as future work.}

%%% Current state of the art
% Failing test cases
Presenting students with \emph{failing test cases} still remains one of the most
common approaches for providing feedback on erroneous programming assignment
attempts~\cite{FIXME,FIXME}. These test cases can be carefully selected by instructors to guide
their students in debugging their code. However, for students that are not yet
experienced in programming, such feedback may not be
sufficient~\cite{FIXME}. We propose to provide
more targeted feedback highlighting both possible program
locations that are responsible for the program errors as well as possible
repairs that will lead the student to a correct solution.

% Fault localization FIXME: make this better
Recent research~\citep[][]{Seidel:2017, Zhang2014-lv} has proposed the use of
\emph{fault localization} for more guided student feedback. Fault localization is the
task of identifying possible program elements (\eg lines, expressions \etc) that
are the cause of the program error at hand. While these techniques can pinpoint,
with high accuracy, the precise location in the student's code that causes the
error and needs to be fixed, inexperienced programmers may not know how to act on
such feedback~\cite{FIXME}. We hypothesize that fault localization combined with
\emph{program repair}
can provide the appropriate feedback for these novice programmers.

% Program repair
Recent automated systems for providing student feedback via program repair
demand significant manual effort from the instructor to be
effective~\cite{FIXME}. For example, the AutoGrader system requires a
reference solution per programming assignment and a custom error model that
will help
repair incorrect student attempts. The custom error model represents a
significant annotation burden for the instructor, and the repair generation
itself can be time-consuming due to constraint solving, which can hamper
the effectiveness of online feedback~\cite{FIXME}.
As a second example, systems like CLARA [TODO: cite] use
partitioning and machine learning to avoid much instructor annotation effort
and produce repairs relatively quickly, but can
lead to
imprecise and non-minimal repairs~\cite{FIXME}.
Moreover, their use of Integer Linear
Programming (ILP) for variable alignment between possible local repairs can hurt
its scalability \WRW{notes: you just said CLARA can produce repairs
``relatively quickly'' one sentence ago: we cannot claim that CLARA is both
quick and also non-scalable}.
% [TODO: Section N] provides a detailed survey of related work.
% TODO: Add Seminal(it's the one on our human study) here?
% TODO: Maybe add another tool?

% FIXME: help here!!
%%% Our technique
\paragraph{Data-Driven Program Repair.}
In this paper, we introduce \toolname, a \emph{data-driven} approach to
repairing novice attempts on programming assignments based on supervised
learning. \toolname uses a large dataset containing pairs of ill-typed
programs \WRW{notes: ill-typed is first used here --- are we only doing
type errors?} and
their subsequent fixes to \emph{automatically learn models} that localize
errors and suggest template fixes \WRW{notes: we should be talking about
the data-driven solution and form of these models much earlier when we list
our insights --- also, let's talk about your dynamic solution of using
interaction traces}, and uses
\emph{program synthesis} to turn those
templates into \emph{program repairs} \WRW{notes: we should hint at this
earlier}. Given a new ill-typed program, \toolname generates a list of
potential solutions ranked by likelihood and an \emph{edit-distance}
metric. We evaluate \toolname by comparing its accuracy on a set of over
4,500 ill-typed \ocaml programs drawn from two instances of an introductory
programming course \WRW{notes: this is an amazing evaluation! we should be
mentioning it earlier and in a little more detail}.


% 1. ``We identify an important problem in the world.'' Be more specific than
% just ``bugs''. Are we focusing on novices and students or are we focusing
% on general software defects? Are we focusing on strongly-typed functional
% languages or are we proposing something generic? What ``news article'' or
% ``survey paper'' citations can you list here to convince me that this is a
% big deal?

% 2. ``Here are the properties that a good solution must have.'' Pick three.
% Here are some examples: must be applicable to students; must produce
% answers quickly; must produce answers that are very close to what humans
% would do; must apply to programs from a wide range of application domains.

% 3. ``Here is the current state of the art. Note that each of these fails to
% obtain at least one of the properties above.'' Candidates: manual debugging
% (bad because of X and Y); using something like genprog (bad because of X
% and Z); using pure fault localization (bad because of A and B); using delta
% debugging or git/svn blame (bad because of P and Q); using something like
% Nate or Sherrloc (bad because of Q and R).

% 4. ``Here are our two or three insights. These insights are the
% underpinning of our solution.'' What are the most important ones?
% Candidates: blame-labeled training sets are available; student repairs fall
% into a reasonable number of categories (admitting a classification
% technique); program repair can be viewed as a synthesis problem;
% generalized ASTs can handle typed and untyped program manipulations.

% 5. ``We combine those insights into TECHNIQUE. It works by steps A, B and
% C, which allow it to obtain the properties P1, P2, and P3 of a good
% solution.'' Briefly condense the steps from Section 2 here.

% 6. ``We evaluate our technique. For Property P1, we use metric M1 and must
% be at least as good as S1 to be successful. For Property P2, we use metric
% M2 and must be at least as good as S2 to be successful. We obtain property
% P3 by construction.'' Fill in the blanks. In addition, for every benchmark
% set or human study used, indicate why you are sampling correctly --- why
% those results are likely to generalize.

% The contributions of this paper are as follows:
% \begin{itemize}
%   \item The algorithm. FIXME.
%   \item The dataset. FIXME --- is this a contribution?
%   \item The empirical evaluation. FIXME.
%   \item The human study. FIXME.
% \end{itemize}
