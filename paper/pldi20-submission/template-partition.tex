\section{Learning Fix Templates}
\label{sec:templ-partition}

We start by introducing our approach for extracting useful \emph{fix templates}
from a training dataset comprised of paired erroneous and fixed programs.
%
We express those templates in terms of a language that allows us to succinctly
represent fixes in a way that captures the essential structure of various fix
patterns that novices use in practice.
%
However, extracting a single fix template for \emph{each} fix in the program
pair dataset yields too many templates to perform accurate predictions.
%
Hence, we define a \emph{similarity} relation between templates, which we use to
\emph{partition} the extracted templates into a smaller set, that will make it
easier to train precise models to predict fixes.

\begin{figure}
\small
\centering
\begin{minipage}[c]{\linewidth}
  \[
  \boxed{
  \begin{array}{rcl}
  e & ::=    & x \spmid \efun{x}{e} \spmid \eapp{e}{e} \spmid \elet{x}{e}{e} \\
    & \spmid & n \spmid \eplus{e}{e}\\
    & \spmid & b \spmid \eif{e}{e}{e} \\
    & \spmid & \epair{e}{e} \spmid \epcase{e}{x}{x}{e} \\
    & \spmid & \enil \spmid \econs{e}{e} \spmid \ecase{e}{e}{x}{x}{e} \\[0.05in]

  n & ::= &  0, 1, -1, \ldots \\[0.05in]

  b & ::= &  \etrue \spmid \efalse \\[0.05in]

  t & ::= & \alpha \spmid \tbool \spmid \tint \spmid \tfun{t}{t} \spmid \tprod{t}{t} \spmid \tlist{t} \\[0.05in]
  \end{array}
  }
  \]
  \captionof{figure}{Syntax of \repairLang}
  \label{fig:syntax}
\end{minipage}
\begin{minipage}[c]{\linewidth}
  \lstDeleteShortInline{|} % sigh...
  \[
  \boxed{
  \begin{array}{lcl}
    \V           & \defeq & \List{\R}\\
    \Runit       & \defeq & \{r \in \R\ |\ 0 \le r \le 1\} \\ % [0,1]\\
    \T           & \defeq & e\\
    \featuresym  & : & \List{e \to \R} \\
    \labelsym    & : & e \times e \to \List{e} \\
    \extractTsym & : & \featuresym \to e \times e \to \List{\V \times \List{\T}} \\
    \trainTsym   & : & \List{\V \times \List{\T}} \to \Model \\
    \predictTsym & : & \Model \to \V \to \List{\Runit} \\
    \midrule
    \repairsym   & : & \Model \to e \to \List{e \times \Runit \times \List{\Runit}}
  \end{array}
  }
  \]
  \lstMakeShortInline{|}
  \captionof{figure}{
    A high-level API for converting program pairs to
    feature vectors and template labels.
  }
  \label{fig:api}
\end{minipage}
\end{figure}


\subsection{Representing User Fixes}
\label{sec:templ-partition:lang}

\paragraph{Repair Template Language.}
\autoref{fig:rtl-syntax} describes our Repair Template Language, \repairLang,
which is a lambda calculus with integers, booleans, pairs, and lists, that
extends our core ML language \lang (\autoref{fig:ml-syntax}) with several
syntactic abstraction forms:

\begin{enumerate}
    \item \emph{Abstract variable} names $\hat{x}$  are used to denote variable
    occurrences for functions, variables and binders, \ie $\hat{x}$ denotes
    an unknown variable name in \repairLang;

    \item \emph{Abstract literal} values $\hat{n}$ can represent \emph{any}
    integer, floating point number, boolean value, character, or string;

    \item \emph{Abstract operators} $\oplus$ similarly denote unknown unary or
    binary operators;

    \item \emph{Wildcard} expressions $\_$ are used to represent \emph{any}
    expression in \repairLang, \ie a program \emph{hole}.
\end{enumerate}

Recall from \autoref{sec:overview:templates} that we define fixes as
replacements of expressions with new candidate expressions at specific program
locations. Therefore, we use candidate expressions over \repairLang to represent
fix templates.

\paragraph{Generalizing ASTs.}
A \emph{Generic Abstract Syntax Tree} (GAST) is a term from \repairLang that
represents many possible expressions from \lang. GASTs are abstract from
standard ASTs over the core language \lang using the $\abstrsym$ function that
takes as input an expression $e^{ML}$ over \lang and a depth $d$ and returns an
expression $e^{RTL}$ over \repairLang, \ie a GAST with all variables, literals
and operators of $e^{ML}$ abstracted and all subexpressions starting at depth
greater than $d$ pruned and replaced with holes $\_$.

\paragraph{Example.}
Recall our example program \mbd in \autoref{fig:mulByDigit}.
%
The expression |[hd * i]| replaces |(hd * i)| in line 4, and hence, is the
user's \emph{fix}, whose AST is given in \autoref{fig:fix_ast}.
%
The output of $\abstrsym$, given this AST and a depth $d = 2$ as input, would be
the GAST in \autoref{fig:templ_gast}, where the operator |*| has been replaced
with an abstract operator $\oplus$, and the sub-terms |hd| and |i| at depth 2
have been abstracted to wildcard expressions $\_$.
%
Hence, the \repairLang term |[_ $\oplus$ _]| represents a potential fix template
for \mbd.

\begin{figure}
    \centering
    \begin{minipage}[c]{0.49\linewidth}
        \centering
        \begin{tikzpicture}
        [font=\small, edge from parent,
        every node/.style={top color=white, bottom color=black!15,
        circle, minimum size=6mm, draw=black!75,
        very thick, drop shadow, align=center},
        edge from parent/.style={draw=black!75,thick},
        level distance=1.0cm]
        \node (cons) {\texttt{::}}
            child { node (mult) {\texttt{*}}
                child { node {\texttt{hd}}}
                child { node {\texttt{i}}}
                }
            child {node (elist) {\texttt{[]}}};
        \end{tikzpicture}
        \subcaption{Fix AST}
        \label{fig:fix_ast}
    \end{minipage}
    \begin{minipage}[c]{0.49\linewidth}
        \centering
        \begin{tikzpicture}
            [font=\small, edge from parent,
            every node/.style={top color=white, bottom color=black!15,
            circle, minimum size=6mm, draw=black!75,
            very thick, drop shadow, align=center},
            edge from parent/.style={draw=black!75,thick},
            level distance=1.0cm]
            \node (cons) {\texttt{::}}
                child { node (mult) {\texttt{$\oplus$}}
                    child { node {\texttt{$\_$}}}
                    child { node {\texttt{$\_$}}}
                    }
                child {node (elist) {\texttt{[]}}};
            \end{tikzpicture}
        \subcaption{Template GAST}
        \label{fig:templ_gast}
    \end{minipage}
    \caption{(left) The fix from example \autoref{fig:mulByDigit} and (right) a possible template for that fix.}
\end{figure}


\subsection{Extracting Fix Templates from a Dataset}
\label{sec:templ-partition:templates}

Our approach fully automates the extraction of fixes by harvesting a set of fix
templates from a training set of program pairs.
%
Given a program pair $(\pbad, \pfix)$ from the dataset, we extract a unique fix
for each location in $\pbad$ that changed in $\pfix$.
%
We do so with an expression-level $\diffsym$~\citep{Lempsink2009-xf} function.
%
Recall that we consider fixes to be replacements of expressions, and
therefore we abstract these extracted changes as our fix templates.

\paragraph{Contextual Repairs.}
%
Following \cite{Felleisen92} let $\econtext{}$ be the \emph{context} in which an
expression $e$ appears in a program $p$, \ie the program $p$ with $e$
replaced by a hole $\_$.
%
We write that $p = \context{}{e}$, meaning that if we fill the hole with the
original expression $e$ we obtain the original program $p$.
%
In this fashion, $\diffsym$ finds a \emph{minimal} (in number of nodes)
expression replacement $\efix$ for an expression $\ebad$ in $\pbad$, such that
$\pbad = \context{\pbad}{\ebad}$ and $\context{\pbad}{\efix} = \pfix$.
%
There may be several such expressions in a program, and $\diffsym$ returns all
such changes.

\paragraph{Examples.} If $\eapp{f}{x}$ is rewritten to $\eapp{g}{x}$, the context is
$\econtext{} = \eapp{\_}{x}$ and the fix is $g$, since $\context{}{g} = \eapp{g}{x}$.

If $\eapp{f}{x}$ is rewritten to $\eplus{(\eapp{f}{x})}{1}$, the context is
$\econtext{} = \_$, and the fix is the whole expression
$\eplus{(\eapp{f}{x})}{1}$, thus $\context{}{\eplus{(\eapp{f}{x})}{1}} =
\eplus{(\eapp{f}{x})}{1}$. (Even though $\eapp{f}{x}$ appears in
both the original and fixed programs, we consider the application expression
$\eapp{f}{x}$ --- but not $f$ or $x$ --- to be replaced with the $+$ operator.)

\subsection{Partitioning the Templates}

Programs over \lang force similar fixes, such as changes to variable names, to
have identical GASTs. Our next step is to define a notion of program fix
\emph{similarity}. Our definition supports the formation of a small but widely
applicable set of fix templates. This small set is used to train a repair
predictor.

\label{subsec:partitioning}
% \begin{figure*}
% \begin{minipage}{\textwidth}
% \begin{haskellcode}
% ==data Expr== = Var | Bop Expr Expr | App [Expr] | ...

% ==sim :: Expr -> Expr -> Bool==
% sim Var         Var         = True
% sim (Bop x1 y1) (Bop x2 y2) = sim [x1, y1] [x2, y2] ==||== sim [x1, y1] [y2, x2]
% sim (App xs)    (App ys)    = any (\ys' -> any (\xs' -> sim xs' ys') xss) yss
%     where
%         xss = permutations xs
%         yss = permutations ys
% sim _           _           = False

% ==sim :: [Expr] -> [Expr] -> Bool==
% sim (x:xs) (y:ys) = sim x y && sim xs ys
% \end{haskellcode}
% \end{minipage}
% \caption{$\simil{e_1}{e_2}$ denotes when the GAST $e_1$ is similar to $e_2$.}
% \label{fig:similar}
% \end{figure*}

\paragraph{GAST Similarity.}
% %
% \autoref{fig:similar} formalizes a relation that states when
% an expression $e_1$ is \emph{similar to} $e_2$  (written \simil{e_1}{e_2}).
% %
Two GASTs are \emph{similar} when
the root nodes are the same and their child subtrees (if any) can be ordered
such that they are pairwise similar. For example, $x + 3$ and $7 - y$ yield
\emph{similar} GASTs where the root nodes are both abstract binary operators,
one child is an abstract literal, and one child is an abstract variable.
        % \eg $e_{11} \oplus e_{12}$ and $e_{21} \oplus e_{22}$ are
        % similar \textit{iff} $(e_{11}, e_{21})$ and $(e_{12}, e_{22})$ or
        % $(e_{11}, e_{22})$ and $(e_{12}, e_{21})$ are pair-wise similar.

% TODO: partition is novel? compare to previous work
\paragraph{Partitioning.}
GAST similarity defines a relation which is reflexive, symmetric, and transitive
and thus an \emph{equivalence} relation. We can now define \emph{partitioning}
as the computation of all possible \emph{equivalence classes} of our extracted
fix templates \wrt GAST similarity. Each class can consist of several
member-expressions and any one of them can be viewed as the class
\emph{representative}. Each representative can then be used as a fix template to
produce repairs for ill-typed programs.

For example, $\hat{x} \oplus \hat{n}$ and $\hat{n} \oplus \hat{x}$ are similar
so they are in the same class. Either one can be used as the representative and
our repair algorithm in \autoref{sec:synthesis} will essentially consider both
when fixing an erroneous program with this template.

Finally, our partitioning algorithm returns the top $N$ equivalence classes
based on their member-expressions frequency in the dataset. $N$ is a parameter
of the algorithm and is chosen to be as small as possible while the top $N$
classes represent a large enough portion of the dataset. We discuss more of its
value in \autoref{sec:eval}.
