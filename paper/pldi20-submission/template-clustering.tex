\section{Learning Fix Templates}
\label{sec:templ-cluster}

In this section, we introduce our approach for extracting useful fix schemas ---
or \emph{fix templates} --- from a given dataset of already fixed programs. We
represent those templates in terms of a simple language \repairLang
(\autoref{fig:rtl-syntax}), a lambda calculus with integers, booleans, pairs,
and lists, but also with some abstractions that allow us to represent more
general abstract fix schemas, that simultaneously capture the core structure of
various fix patterns that novices use.

% A template over \repairLang is a \emph{Generic Abstract Syntax Tree (GAST)}.
% Those GASTs will then be used to predict and generate possible repairs for new
% erroneous programs, as described in the next two sections. However,
Having a single fix template for each program pair that appears in the dataset
can lead to a large number of templates. Therefore, we define  a
\emph{similarity relation} between fix templates, that will enable us to
\emph{cluster} the existing extracted fix templates into a smaller set, making
it possible to train predictive models that can precisely suggest fixes.
% Therefore, we define a \emph{GAST similarity metric} that will enable us
% classify fix patterns that don't have the exact same syntax tree but appear to
% be of the same nature, to be represented by the same fix template. For example,
% changing different arguments of a function application should be represented by
% the same fix template.

\begin{figure}
\small
\centering
  \begin{minipage}[c]{\linewidth}
  \[
  \boxed{
  \begin{array}{rcl}
  e & ::=    & x \spmid \efun{x}{e} \spmid \eapp{e}{\bar{e}} \spmid \elet{x}{e}{e} \\
    & \spmid & n \spmid b \spmid \eplus{e}{e} \spmid \eif{e}{e}{e} \\
    & \spmid & \epair{e}{e} \spmid \epcase{e}{x}{x}{e} \\
    & \spmid & \enil \spmid \econs{e}{e} \spmid \ecase{e}{e}{x}{x}{e} \\[0.05in]

  n & ::= &  0, 1, -1, \ldots \\[0.05in]

  b & ::= &  \etrue \spmid \efalse \\[0.05in]

  t & ::= & \alpha \spmid \tbool \spmid \tint \spmid \tfun{t}{t} \spmid \tprod{t}{t} \spmid \tlist{t} \\[0.05in]
  \end{array}
  }
  \]
  \captionof{figure}{Syntax of \lang}
  \label{fig:ml-syntax}
  \end{minipage}
  \begin{minipage}[c]{\linewidth}
    \[
    \boxed{
    \begin{array}{rcl}
    e & ::=    & \_  \spmid \hat{x} \spmid \efun{\hat{x}}{e} \spmid \eapp{\hat{x}}{\bar{e}} \spmid \elet{\hat{x}}{e}{e} \\
      & \spmid & \hat{n} \spmid \ebop{e}{e} \spmid \eif{e}{e}{e} \\
      & \spmid & \epair{e}{e} \spmid \epcase{e}{\hat{x}}{\hat{x}}{e} \\
      & \spmid & \enil \spmid \econs{e}{e} \spmid \ecase{e}{e}{\hat{x}}{\hat{x}}{e} \\[0.05in]
    \end{array}
    }
    \]
    \captionof{figure}{Syntax of \repairLang}
    \label{fig:rtl-syntax}
  \end{minipage}
\end{figure}




\subsection{Representing User Fixes}
\label{sec:templ-cluster:lang}
Here, we define \repairLang and we describe how we can use it to represent fixes
from a dataset of program pairs.

\paragraph{Repair Template Language.}
We define here our Repair Template Language (RTL) as a simple lambda calculus
(\repairLang syntax in \autoref{fig:rtl-syntax}) with integers, booleans, pairs,
and lists. Our RTL contains all the expressions that a core ML language \lang
(syntax in \autoref{fig:ml-syntax}) would include, but with four key
differences.

\begin{enumerate}
    \item Variable names $\hat{x}$ used for functions, variables and patterns
    are left \emph{unspecified}. That means that $\hat{x}$ denotes a unknown
    variable name in \repairLang.
    \item Literal values $\hat{n}$ can represent any integer or floating point
    number, boolean value or character and string. Like variables, these are
    left unspecified.
    \item Operators $\oplus$ are also left unspecified.
    \item A \emph{wildcard} expression $\_$ is added, which is used to represent
    \emph{any} expression in \repairLang, \ie a program \emph{hole}.
\end{enumerate}

Recall from \autoref{sec:overview:templates} that we define fixes as
replacements of expressions with new candidate expressions at specific program
locations. Therefore, a fix template $\T$ will represent such candidate
expressions over \repairLang.

\paragraph{Generic ASTs.}
We now define \emph{Generic ASTs} (GASTs), \ie abstract syntax trees
representing expressions over \repairLang which have some abstracted details,
like variable names or numerical values. GASTs are extracted from regular ASTs
over the core language \lang using the $\abstrsym$ function. The $\abstrsym$
function takes as input an expression $e^{ML}$ over \lang and a depth $d$ and
returns an expression $e^{RTL}$ over \repairLang, \ie a GAST with all variables,
literals and operators of $e^{ML}$ abstracted and all subexpressions starting at
depth greater than $d$ pruned and replaced with holes $\_$.
\begin{figure}
    \centering
    \begin{minipage}[c]{0.49\linewidth}
        \centering
        \begin{tikzpicture}
        [font=\small, edge from parent,
        every node/.style={top color=white, bottom color=black!15,
        circle, minimum size=6mm, draw=black!75,
        very thick, drop shadow, align=center},
        edge from parent/.style={draw=black!75,thick},
        level distance=1.3cm]
        \node (cons) {\texttt{::}}
            child { node (mult) {\texttt{*}}
                child { node {\texttt{hd}}}
                child { node {\texttt{i}}}
                }
            child {node (elist) {\texttt{[]}}};
        \end{tikzpicture}
        \subcaption{Fix AST}
        \label{fig:fix_ast}
    \end{minipage}
    \begin{minipage}[c]{0.49\linewidth}
        \centering
        \begin{tikzpicture}
            [font=\small, edge from parent,
            every node/.style={top color=white, bottom color=black!15,
            circle, minimum size=6mm, draw=black!75,
            very thick, drop shadow, align=center},
            edge from parent/.style={draw=black!75,thick},
            level distance=1.3cm]
            \node (cons) {\texttt{::}}
                child { node (mult) {\texttt{$\oplus$}}
                    child { node {\texttt{$\_$}}}
                    child { node {\texttt{$\_$}}}
                    }
                child {node (elist) {\texttt{[]}}};
            \end{tikzpicture}
        \subcaption{Template GAST}
        \label{fig:templ_gast}
    \end{minipage}
    \caption{(left) The fix from example \autoref{fig:mulByDigit} and (right) a possible template for that fix. \GS{does this help? maybe move this to overview?}}
\end{figure}


Recall our example program \mbd at \autoref{fig:mulByDigit}. The expression
|[hd * i]| replaces |(hd * i)| in line 4, thus it is the user's fix. Its AST
is given in \autoref{fig:fix_ast}. The output of $\abstrsym$, given this AST and
a depth $d = 1$ as input, would be the GAST in \autoref{fig:templ_gast}. This
can also be a potential fix template for \mbd. We discuss more about fix
templates next.



\subsection{Extracting Fix Templates from a Dataset}
\label{sec:templ-cluster:templates}
We now show how we acquire fix templates from a dataset of program pairs.

\paragraph{Extracting Templates.}
At least one fix template is extracted from each program pair in the dataset.
For each program pair $(\pbad, \pfix)$ in the dataset, we extract a unique fix
for each location in $\pbad$ that changed in $\pfix$. We achieve that with an
expression-level $\diffsym$~\citep{Lempsink2009-xf} function, which extracts all
changes between a program pair $(\pbad, \pfix)$. Recall again that we consider
fixes to be replacements of expressions, and therefore we use the extracted
changes as our fix templates.

Let $\econtext{}$ be the \emph{context} that an expression $e$ appears in a
program $p$, \ie the program $p$ with expression $e$ replaced with a hole $\_$.
We write that $p = \context{}{e}$, meaning that if we fill the hole with the
original expression $e$ we get the original program $p$. In this fashion, $\diffsym$ finds
a \emph{minimal} (in number of nodes) expression replacement $\efix$ for an
expression $\ebad$ in $\pbad$, such that $\pbad = \context{\pbad}{\ebad}$ and
$\context{\pbad}{\efix} = \pfix$. Obviously, there may be several such
expressions in a program, and $\diffsym$ returns all such changes between the
erroneous program $\pbad$ and its fixed counterpart $\pfix$.

For example, if an expression has been replaced wholesale with another
expression, \eg if $\eapp{f}{x}$ is rewritten to $\eapp{g}{x}$, we will have
that the context is $\econtext{} = \eapp{\_}{x}$, where $\_$ represents the
hole, and then the fix will be $g$, since $\context{}{g} = \eapp{g}{x}$. In
another example, if a new expression has been inserted \emph{around} an existing
expression, \eg if $\eapp{f}{x}$ is rewritten to $\eplus{(\eapp{f}{x})}{1}$, the
context will be more general where $\econtext{} = \_$, since we consider the
application expression $\eapp{f}{x}$ (but not $f$ or $x$) to be replaced with
the $+$ operator, and therefore the fix will be the whole expression, thus
$\context{}{\eplus{(\eapp{f}{x})}{1}} = \eplus{(\eapp{f}{x})}{1}$.
% TODO: maybe talk about the drawback of the last choice

% Using the $\diffsym$ function first, we acquire the changed expressions of the
% dataset. These changes are potentially the templates we are going to use.
% However, they contain a lot of \emph{local} information to the specific program
% that they were extracted from. So $\templatesym$ takes the changed expressions
% defined in \lang and \emph{transforms} them into expressions in \repairLang,
% essentially making them into templates $\T$. Those templates are represented as
% GASTs and are pruned at a pre-defined depth $d$. The value of $d$ can be as low
% as 1, but to capture more structure to our templates, slightly bigger values
% would work better. We then see in \autoref{subsec:clustering} how we can group
% together all these templates.

% We identify the fixes for each ill-typed program with an expression-level
% $\diffsym$~\citep{Lempsink2009-xf}. We consider two sources of changes.


\subsection{Clustering the Templates}
We here show how we cluster the possibly vast number of fix templates acquired
from each program pair in a dataset using a \emph{similarity relation} between
the fix templates' GASTs.
\label{subsec:clustering}
\begin{figure*}
    \begin{minipage}{\textwidth}
        \begin{lstlisting}[language=haskell, frame=single]
        data Expr
            = Var
            | App Expr [Expr]
            | Bop Op Expr Expr
            | ...
            | Hole

        similar :: Expr -> Expr -> Bool
        similar (Var v1) (Var v2) = v1 == v2
        similar (App e1 es1) (App e2 es2)
            = and $ zipWith similar (e1 ++ es1) (e2 ++ es2)
        similar (Bop b1 e11 e12) (Bop b2 e21 e22)
            = b1 == b2 && similar e11 e21 && similar e12 e22
        similar ...
        similar Hole e = True
        similar e Hole = True
        similar Hole Hole = True
        similar _ _ = False

        \end{lstlisting}
    \end{minipage}
    \caption{\GS{Just for refence for similarity relation. It is a relation but not transitive. For example, $x + \_$ and $(x + y) + \_$ are both similar to $\_ + \_$, but they are not similar to each other. However, $\_ + \_$ will be chosen as the cluster representative (fix template) and all three will be in the same cluster.}}
    \label{fig:haskell}
\end{figure*}

% TODO: Should we give a better name than "similarity relation"
\paragraph{GAST Similarity Relation.}
Having programs written over \repairLang forces similar fixes, \ie changing a
variable name, to have identical GASTs. We want to extend and generalize this
``similarity'' of program fixes, that will in turn enable us to cluster all the
gathered fixes into a small but generally applicable number of fix templates.

Therefore, two expressions over \repairLang are \emph{similar} when at least one
of the following rules hold \GS{Look code in \autoref{fig:haskell}}:
\begin{enumerate}
    \item Two non-terminal expressions over \repairLang are similar when the
    top-level non-terminal is the same and there is a \emph{total} bijective
    relation between their children expressions, such that \emph{all} the
    related expressions are themselves similar, \eg two binary operator
    expressions are the same if their operands are similar.
    \item A wildcard expression is similar with any possible expression over
    \repairLang.
    \item A terminal expression over \repairLang is similar only with another
    terminal expression of the same kind, \eg two variables are similar.
\end{enumerate}

\paragraph{Clustering.}
The main goal of clustering is to make our predictive models more scalable
and applicable to more programs by avoiding assigning different templates
to equivalent expressions. We define
clustering as the task of grouping together similar expressions over \repairLang
such that each group can then be used as a fix template to produce repairs for
ill-typed programs. Each group can consist of several \emph{member-expressions}
and each one of them can be used as the cluster \emph{representative}.

The clustering algorithm uses the extracted fix templates and groups together
the templates that are similar based on our ealrier definition of expression
similarity. It then returns the Top-$N$ based on their popularity on their
training set. $N$ is considered a parameter of algorithm that can be defined
prior the training and clustering process, and is usually around 30 to 50 [TODO:
cite similar template papers].

TODO: give an example
