\section{Learning Fix Templates}
\label{sec:templ-cluster}

We start by introducing our approach for extracting useful \emph{fix templates}
--- from a training dataset of comprising erroneous-and-fixed programs.
%
We represent those templates in terms of a language that allow us to succinctly
represent fixes in way that captures the essential structure of various fix
patterns that novices use in practice.
%
Having a single fix template for \emph{each} program pair in the dataset yields
too many templates to perform accurate prediction.
%
Hence, we define a \emph{similarity} relation between templates, to let us
\emph{partition} the extracted templates into a smaller set that makes it
possible to train precise predictive models to predict fixes.

\begin{figure}
\small
\centering
\begin{minipage}[c]{\linewidth}
  \[
  \boxed{
  \begin{array}{rcl}
  e & ::=    & x \spmid \efun{x}{e} \spmid \eapp{e}{e} \spmid \elet{x}{e}{e} \\
    & \spmid & n \spmid \eplus{e}{e}\\
    & \spmid & b \spmid \eif{e}{e}{e} \\
    & \spmid & \epair{e}{e} \spmid \epcase{e}{x}{x}{e} \\
    & \spmid & \enil \spmid \econs{e}{e} \spmid \ecase{e}{e}{x}{x}{e} \\[0.05in]

  n & ::= &  0, 1, -1, \ldots \\[0.05in]

  b & ::= &  \etrue \spmid \efalse \\[0.05in]

  t & ::= & \alpha \spmid \tbool \spmid \tint \spmid \tfun{t}{t} \spmid \tprod{t}{t} \spmid \tlist{t} \\[0.05in]
  \end{array}
  }
  \]
  \captionof{figure}{Syntax of \repairLang}
  \label{fig:syntax}
\end{minipage}
\begin{minipage}[c]{\linewidth}
  \lstDeleteShortInline{|} % sigh...
  \[
  \boxed{
  \begin{array}{lcl}
    \V           & \defeq & \List{\R}\\
    \Runit       & \defeq & \{r \in \R\ |\ 0 \le r \le 1\} \\ % [0,1]\\
    \T           & \defeq & e\\
    \featuresym  & : & \List{e \to \R} \\
    \labelsym    & : & e \times e \to \List{e} \\
    \extractTsym & : & \featuresym \to e \times e \to \List{\V \times \List{\T}} \\
    \trainTsym   & : & \List{\V \times \List{\T}} \to \Model \\
    \predictTsym & : & \Model \to \V \to \List{\Runit} \\
    \midrule
    \repairsym   & : & \Model \to e \to \List{e \times \Runit \times \List{\Runit}}
  \end{array}
  }
  \]
  \lstMakeShortInline{|}
  \captionof{figure}{
    A high-level API for converting program pairs to
    feature vectors and template labels.
  }
  \label{fig:api}
\end{minipage}
\end{figure}


\subsection{Representing User Fixes}
\label{sec:templ-cluster:lang}

\paragraph{Repair Template Language.}
\autoref{fig:rtl-syntax} describes our Repair Template Language (RTL) which is a
lambda calculus with integers, booleans, pairs, and lists, that extends our core
ML language \lang (\autoref{fig:ml-syntax}) with several syntactic abstraction
forms:

\begin{enumerate}
    \item \emph{Abstract variable} names $\hat{x}$  are used to denote variable
    occurences for for functions, variables and binders, \ie $\hat{x}$ denotes
    an unknown variable name in \repairLang;

    \item \emph{Abstract literal} values $\hat{n}$ can represent \emph{any}
    integer or floating point number, boolean value or character and string;

    \item \emph{Abstract operators} $\oplus$ similarly denote unknown unary or
    binary operators;

    \item \emph{Wildcard} expressions $\_$ are used to represent \emph{any}
    expression in \repairLang, \ie a program \emph{hole}.
\end{enumerate}

Recall from \autoref{sec:overview:templates} that we define fixes as
replacements of expressions with new candidate expressions at specific program
locations. Therefore, a fix template $\T$ will represent such candidate
expressions over \repairLang.

\paragraph{Generalizing ASTs.}
A \emph{Generic Abstract Syntax Tree} (GAST) is a term from \repairLang that
represents many possible expressions from \lang. GASTs are extracted from
regular terms (ASTs) over the core language \lang using the $\abstrsym$ function
that takes as input an expression $e^{ML}$ over \lang and a depth $d$ and
returns an expression $e^{RTL}$ over \repairLang, \ie a GAST with all variables,
literals and operators of $e^{ML}$ abstracted and all subexpressions starting at
depth greater than $d$ pruned and replaced with holes $\_$.

\begin{figure}
    \centering
    \begin{minipage}[c]{0.49\linewidth}
        \centering
        \begin{tikzpicture}
        [font=\small, edge from parent,
        every node/.style={top color=white, bottom color=black!15,
        circle, minimum size=6mm, draw=black!75,
        very thick, drop shadow, align=center},
        edge from parent/.style={draw=black!75,thick},
        level distance=1.0cm]
        \node (cons) {\texttt{::}}
            child { node (mult) {\texttt{*}}
                child { node {\texttt{hd}}}
                child { node {\texttt{i}}}
                }
            child {node (elist) {\texttt{[]}}};
        \end{tikzpicture}
        \subcaption{Fix AST}
        \label{fig:fix_ast}
    \end{minipage}
    \begin{minipage}[c]{0.49\linewidth}
        \centering
        \begin{tikzpicture}
            [font=\small, edge from parent,
            every node/.style={top color=white, bottom color=black!15,
            circle, minimum size=6mm, draw=black!75,
            very thick, drop shadow, align=center},
            edge from parent/.style={draw=black!75,thick},
            level distance=1.0cm]
            \node (cons) {\texttt{::}}
                child { node (mult) {\texttt{$\oplus$}}
                    child { node {\texttt{$\_$}}}
                    child { node {\texttt{$\_$}}}
                    }
                child {node (elist) {\texttt{[]}}};
            \end{tikzpicture}
        \subcaption{Template GAST}
        \label{fig:templ_gast}
    \end{minipage}
    \caption{(left) The fix from example \autoref{fig:mulByDigit} and (right) a possible template for that fix.}
\end{figure}


\paragraph{Example.}
Recall our example program \mbd at \autoref{fig:mulByDigit}.
%
The expression |[hd * i]| replaces |(hd * i)| in line 4, and hence, is the
user's \emph{fix}, whose AST is given in \autoref{fig:fix_ast}.
%
The output of $\abstrsym$, given this AST and a depth $d = 1$ as input, would be
the GAST in \autoref{fig:templ_gast}, where the operator |*| has been replaced
with an abstract operator, and where the sub-terms |hd| and |i| at depth 2 have
been abstracted to wildcard expressions $\_$.
%
Hence, the \repairLang term |[_ # _]| represents a potential fix template for
\mbd.

\subsection{Extracting Fix Templates from a Dataset}
\label{sec:templ-cluster:templates}

Our approach fully automates the prediction of repairs by harvesting a set of
fix template from a training set of program pairs.
%
Given a program pair $(\pbad, \pfix)$ from the dataset, we extract a unique fix
for each location in $\pbad$ that changed in $\pfix$.
%
We do so with an expression-level $\diffsym$~\citep{Lempsink2009-xf} function,
which extracts all changes between a program pair $(\pbad, \pfix)$.
%
Recall again that we consider fixes to be replacements of expressions, and
therefore we use the extracted changes as our fix templates.

\paragraph{Contextual Repairs.}
%
Following \cite{Felleisen92} let $\econtext{}$ be the \emph{context} that an
expression $e$ appears in a program $p$, \ie the program $p$ with expression $e$
replaced with a hole $\_$.
%
We write that $p = \context{}{e}$, meaning that if we fill the hole with the
original expression $e$ we get the original program $p$.
%
In this fashion, $\diffsym$ finds a \emph{minimal} (in number of nodes)
expression replacement $\efix$ for an expression $\ebad$ in $\pbad$, such that
$\pbad = \context{\pbad}{\ebad}$ and $\context{\pbad}{\efix} = \pfix$.
%
There may be several such expressions in a program, and $\diffsym$ returns all
such changes between the erroneous program $\pbad$ and its fixed counterpart
$\pfix$.

\paragraph{Example.}
%
For example, if a new expression has been inserted \emph{around} an existing
expression, \eg if $\eapp{f}{x}$ is rewritten to $\eapp{g}{x}$, we will have
that the context is $\econtext{} = \eapp{\_}{x}$, where $\_$ represents the
hole, and then the fix will be $g$, since $\context{}{g} = \eapp{g}{x}$.

If instead, an expression has been replaced wholesale with another expression,
\eg if $\eapp{f}{x}$ is rewritten to $\eplus{(\eapp{f}{x})}{1}$, the
\emph{context} will be more general where $\econtext{} = \_$, since we consider
the application expression $\eapp{f}{x}$ (but not $f$ or $x$) to be replaced
with the $+$ operator, and therefore the \emph{fix} will be the whole
expression, thus $\context{}{\eplus{(\eapp{f}{x})}{1}} =
\eplus{(\eapp{f}{x})}{1}$.
% TODO: maybe talk about the drawback of the last choice

\subsection{Clustering the Templates}

Having programs written over \repairLang forces similar fixes, \ie changing a
variable name, to have identical GASTs. We want to further extend and generalize
the notion of ``similarity'' of program fixes, to let us reduce the gathered
fixes into a small but widely applicable set of fix templates, which can then be
used to train a repair predictor.


\label{subsec:clustering}
\begin{figure*}
\begin{minipage}{\textwidth}
\begin{lstlisting}[language=haskell, frame=single]
data Expr = Var | App Expr [Expr] | Bop Op Expr Expr | Hole | ..

sim :: Expr -> Expr -> Bool
sim e              Hole           = True
sim (Var x)        (Var y)        = x == y
sim (App f xs)     (App g ys)     = sim (f:xs) (g:ys)
sim (Bop o1 x1 y1) (Bop o2 x2 y2) = o1 == o2 && sim [x1,x2] [y1, y2]
sim _              _              = False

sim :: [Expr] -> [Expr] -> Bool
sim (e1:e1s) (e2:e2s)             = sim e1 e2 && sim e1s e2s
\end{lstlisting}
\end{minipage}
\caption{$\simil{e_1}{e_2}$ denotes when the GAST $e_1$ is similar to $e_2$.}
\label{fig:similar}
\end{figure*}

\paragraph{GAST Similarity.}
%
\autoref{fig:similar} formalizes a relation that states when
an expression $e_1$ is \emph{similar to} $e_2$  (written \simil{e_1}{e_2}).
%
Intuitevely, $e_1$ is similar to $e_2$ when at least one of the following rules
hold
\begin{enumerate}
    \item every expression is similar to a wildcard $\_$;

    \item the top-level non-terminal is the same their sub-expressions are all
          pair-wise similar, \eg two binary operator expressions are the same if
          their operands are similar;

    \item a terminal expression is similar to another, only when they are the
    same, \eg two variables are similar.
\end{enumerate}
%

\paragraph{Clustering.}
%%
The similarity relation defines a partial order on GAST terms. For example, $x +
\_$ and $(x + y) + \_$ are both similar to $\_ + \_$, but they are not similar
to each other.
%
However, $\_ + \_$ will be chosen as the cluster representative (fix template)
and all three will be in the same cluster.
%
\RJ{this is vague and doesn't make sense; TBD}
%%
The main goal of \emph{clustering} is to make our predictive models scalable and
applicable to more programs by avoiding assigning different templates to
equivalent expressions. We define clustering as the task of grouping together
similar expressions over \repairLang such that each group can then be used as a
fix template to produce repairs for ill-typed programs. Each group can consist
of several \emph{member-expressions} and one of them is the cluster
\emph{representative}.

The clustering algorithm uses the extracted fix templates and groups together
the templates that are similar based on our ealrier definition of expression
similarity. The most generic expression, \ie the expression whose GAST has the
smallest size (which in this context means the one with least number of nodes),
is the cluster representative. Holes $\_$ have a size of $0$, thus they don't
add up to the total size of a GAST.

For example, the following expressions: $\hat{x} \oplus \_$, $\_ \oplus \hat{x}$
and $\_ \oplus \_$ are all similar to one another and therefore will be placed
in the same cluster. However, $\_ \oplus \_$ is the smallest in size and thus
will be the cluster representative.

Finally, our clustering algorithm actually returns the top $N$ clusters based on
their member-expressions frequency in the dataset. $N$ is a parameter of the
algorithm that is chosen in a manner that the top $N$ clusters will
representative a large percentage of the dataset. We discuss more of its value
in \autoref{sec:eval}.
