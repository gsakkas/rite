\section{Template-Guided Repair Synthesis}
\label{sec:synthesis}
We use program synthesis to fully repair a program using predicted fix templates
and locations from our models. We present in
\autoref{sec:synthesis:local-synthesis} a synthesis algorithm for producing
\emph{local repairs} for a given program location. In
\autoref{sec:synthesis:location-rank}, we show how we use local repairs to
repair programs that may also have \emph{multiple} error locations.

\lstMakeShortInline[mathescape=true]{|}

\subsection{Local Synthesis from Templates}
\label{sec:synthesis:local-synthesis}

\mypara{Enumerative Program Synthesis}
We perform a classic \emph{enumerative} program synthesis that is guided by an
fix template. Enumerative synthesis searches all possible expressions over a
language until a high-level specification is reached. In our case, we initially
synthesize independent \emph{local repairs} for a program that already captures
the user's intent. Therefore, the required specification is that the repaired
program is type-safe. However, a stricter specification can also be used, by
only synthesizing programs that have the user's desired type signature, in case
they provide one.

Given a location $l$, a template $t$ and a maximum depth $d$,
\autoref{algo:local-repair-algo} searches over all possible expressions over
\lang that will satisfy those goals by generating a local repair that fills
$t$'s GAST with concrete variables, literals, functions \etc. Our technique can
also reuse subexpressions $e$ at location $l$ for $t$'s concretization to
further optimize the search.

\begin{algorithm}[t]
    \caption{Local Repair Algorithm}
    \label{algo:local-repair-algo}
    \renewcommand{\algorithmicrequire}{\textbf{Input:}}
    \renewcommand{\algorithmicensure}{\textbf{Output:}}
    \begin{algorithmic}[1]
    \Require{Language Grammar \lang, Program $P$, Template $T$, Repair Location $L$, Max Repair Depth $D$}
    \Ensure{Local Repairs $R$}
    \Procedure{Enumerate}{$\lang, P, T, L, D$}
    \State $R \gets \emptyset$
    \ForAll{$d \in [1 \dots D]$}
      \State $\tilde{\alpha} \gets$ \Call{NonTerminalsAt}{$T, d$}
      \ForAll{$\alpha \in$ \Call{RankNonTerminals}{$\tilde{\alpha}, P, L$}}
        \If{\Call{IsHole}{$\tilde{\alpha}$}}
          \State $Q \gets$ \Call{GrammarRules}{$\lang$}
          \State $\tilde{\beta} \gets \{\beta\:|\:(\alpha, \beta) \in Q\}$
          \ForAll{$\beta \in$ \Call{RankRules}{$\tilde{\beta}, T$}}
            \State $\hat{T} \gets$ \Call{ApplyRule}{$T, (\alpha, \beta)$}
          \EndFor
        \Else
          \ForAll{$\sigma \in$ \Call{GetTerminals}{$P, \alpha, \lang$}}
            \State $\hat{T} \gets$ \Call{ReplaceNode}{$T, \alpha, \sigma$}
          \EndFor
        \EndIf
        \State $\hat{P} \gets$ \Call{ReplaceExprAt}{$P, L, \hat{T}$}
        \If{\Call{TypeCheck}{$\hat{P}$}}
          \State $R \gets R \cup \{\hat{P}\}$
        \EndIf
      \EndFor
    \EndFor
    \State \Return{$R$}
    \EndProcedure
    \end{algorithmic}
\end{algorithm}


\mypara{Template-Guided Local Repair}
Using the \textsc{Repair} method (\autoref{algo:local-repair-algo}), we produce
local repairs $R$ for a given location $L$ of an erroneous program $P$.
\textsc{Repair} fills in a template $T$ based on the context-free grammar
$\lang$. It starts from the top-level node of template $T$ and incrementally
traverses downwards the GAST producing candidate local repairs of maximum depth
$D$.

When a hole $\alpha \in T$ is found, the algorithm expands $T$'s GAST one more
level using $\lang$'s production rules $Q$. The production rules are considered
in an ranked order based on the subexpressions that already appear in the rest
of the template $T$ and program location $L$. Each rule is then applied to
template $T$, returning an \emph{instantiated} template $\hat{T}$, which is
inserted into the list of candidate local repairs $R$.

If node $\alpha$ is not a hole, terminals from the subexpressions at location
$L$, the program $P$ in general and the grammar $\lang$ are used to concretize
that node, depending on the $\repairLang$ terminal node $\alpha$. For each of
these template $T$ modifications, we insert an instantiated template $\hat{T}$
into $R$.


\subsection{Ranking Error Locations}
\label{sec:synthesis:location-rank}

\mypara{Error Location Confidence}
Recall from \autoref{sec:templ-pred} that for each subexpression in a program's
type-error slice, our $\Model$ generates a confidence score $\Conf$ for it being
the error location, and our $\ModelT$ scores for each of the selected fix
templates.

Our synthesis algorithm ranks all program locations based on their confidence
<<<<<<< HEAD
scores $\Conf$. For all locations in descending confidence
score order, a fix template is used to produce a repair. Fix templates are also
considered in descending order of confidence. If the algorithm fails to
synthesize a program that type-checks, the next location is considered, until it
succeeds.

\mypara{Multiple Error Locations}
In practice, frequently more than one location needs to be repaired. We thus
extend the above approach to fix programs with multiple error locations.

Let the confidence scores $\Conf$ for all locations in the type error slice from
our error localization model $\Model$ be $(l_1, c_1), \dots, (l_k, c_k)$, where
$l_i$ is a program location and $c_i$ its confidence score. We assume, for
simplicity, that the probabilities $c_i$ are independent. Thus
the probability that \emph{all} the locations $\{l_i \dots l_j\}$ need to be fixed
is the product $c_i \cdots c_j$. Therefore, instead of
ranking and trying to find fixes for single locations $l$,
we use \emph{sets} of locations ($\{l_i\}, \{l_i, l_j\}, \{l_i, l_j, l_k\}$, \etc),
ranked by the products of their confidence scores.
=======
scores $\Conf$. For all locations in descending order of their confidence
scores, a fix template is used to produce a local repair using
\autoref{algo:local-repair-algo}. Fix templates are also considered in
descending order of confidence. Then expressions from the returned list of local
repairs $R$ replace the expression at the given program location, until a
type-correct program is found. The procedure continues with the next template or
error location if it fails for the specific ones.

Following \citep{Lerner2007-dt}, we allow our final local repairs to have
program holes $\_$ or abstracted variable $\hat{x}$ in them. However,
\autoref{algo:local-repair-algo} will prioritize the synthesis of complete
solutions. Abstracted terms over $\repairLang$ are expressions that can have any
type when type-checking concrete solutions, \eg similarly to \ocaml's |raise Exn|.


\mypara{Multiple Error Locations}
In practice, frequently more than one program locations need to be repaired. We
thus extend the above approach to fix programs with multiple errors. Let the
confidence scores $\Conf$ for all locations in the type error slice from our
error localization model $\Model$ be $(l_1, c_1), \dots, (l_k, c_k)$, where
$l_i$ is a program location and $c_i$ its error confidence score. We assume for
simplicity that the probabilities $c_i$ are independent, so the probability that
\emph{all} the locations $\{l_i \dots l_j\}$ need to be fixed is the product
$c_i \cdots c_j$. Therefore, instead of just ranking and trying to find fixes
for single locations $l$, we use \emph{sets} of locations ($\{l_i\}, \{l_i,
l_j\}, \{l_i, l_j, l_k\}$, \etc), ranked by the products of their confidence
scores. Finally, we use \autoref{algo:local-repair-algo} independently for each
location and apply all possible combinations of local repairs to erroneous,
looking again for a type-correct solution.
% In practice we only consider up to \emph{five} locations to be fixed
% simultaneously; any more than that takes too much time to generate and has too
% small a chance of leading to a good solution.


\subsection{Local Synthesis from Templates}
\label{subsec:local-synthesis}

\mypara{Enumerative Program Synthesis}
Our synthesis algorithm is a classic \emph{enumerative} program synthesis method
guided by an input template. Enumerative synthesis searches all possible
expressions over a language until a high-level
specification is reached. In our case, we only try to synthesize a part of the
program that already captures the intent of the user and therefore our only
specification is that the repaired program is type-safe. However, we can also
extend this specification by allowing our algorithm to search for programs that
have the user's desired type signature.

Given a location $l$ and a template $t$, our algorithm searches over all
possible expressions over \lang that will satisfy those goals by generating a
\emph{local repair} that expands $t$'s GAST. One advantage of this technique
is that we can exploit the expression $e$ at location $l$ to further guide our
synthesis, since subexpressions used by the programmer at $l$ are usually reused
for their final repair.

\begin{algorithm}[t]
    \caption{Local Repair Algorithm}
    \label{algo:local-repair-algo}
    \renewcommand{\algorithmicrequire}{\textbf{Input:}}
    \renewcommand{\algorithmicensure}{\textbf{Output:}}
    \begin{algorithmic}[1]
    \Require{Language Grammar \lang, Program $P$, Template $T$, Repair Location $L$, Max Repair Depth $D$}
    \Ensure{Local Repairs $R$}
    \Procedure{Enumerate}{$\lang, P, T, L, D$}
    \State $R \gets \emptyset$
    \ForAll{$d \in [1 \dots D]$}
      \State $\tilde{\alpha} \gets$ \Call{NonTerminalsAt}{$T, d$}
      \ForAll{$\alpha \in$ \Call{RankNonTerminals}{$\tilde{\alpha}, P, L$}}
        \If{\Call{IsHole}{$\tilde{\alpha}$}}
          \State $Q \gets$ \Call{GrammarRules}{$\lang$}
          \State $\tilde{\beta} \gets \{\beta\:|\:(\alpha, \beta) \in Q\}$
          \ForAll{$\beta \in$ \Call{RankRules}{$\tilde{\beta}, T$}}
            \State $\hat{T} \gets$ \Call{ApplyRule}{$T, (\alpha, \beta)$}
          \EndFor
        \Else
          \ForAll{$\sigma \in$ \Call{GetTerminals}{$P, \alpha, \lang$}}
            \State $\hat{T} \gets$ \Call{ReplaceNode}{$T, \alpha, \sigma$}
          \EndFor
        \EndIf
        \State $\hat{P} \gets$ \Call{ReplaceExprAt}{$P, L, \hat{T}$}
        \If{\Call{TypeCheck}{$\hat{P}$}}
          \State $R \gets R \cup \{\hat{P}\}$
        \EndIf
      \EndFor
    \EndFor
    \State \Return{$R$}
    \EndProcedure
    \end{algorithmic}
\end{algorithm}


\mypara{Generating Local Repairs with Templates}
Using our \textsc{Enumerate} method (\autoref{algo:local-repair-algo}), we lazily generate local repairs $R$ for
each subset of locations that is highest in confidence score. The
\textsc{Enumerate} method starts to fill in a template $T$ for location $L$ of
the program $P$ based on the context-free grammar $\lang$. It starts from the
parent node at the first level of the template $T$ and incrementally moves
down the tree. When a hole is found in the tree, the algorithm tries to expand
the tree one more level using $\lang$'s production rules $Q$. The production
rules are considered in an ranked order based on the subexpressions that already
appear in $P$'s location $L$ and the template $T$. It then applies the rule to the
template $T$. If the node $\tilde{\alpha}$ was not a hole, terminals from the
program $P$, the expression $\alpha$ at the location $L$ and the grammar
$\lang$ are used to fill that node, depending on what terminals were used from
$\repairLang$. For example, $\repairLang$'s operator $\oplus$ can be replaced
with $+,\:-,\:\etc$

After one or more rules have been applied to a 
template $T$, we obtain an
\emph{instantiated} template $\hat{T}$, which then replaces the expression at
location $L$. If the new program $\tilde{P}$ type-checks, it is inserted into
the list of generated solutions $R$. $R$ is
generated lazily in practice and the top-N results can be produced for the user. 


% \subsection{Program Repair}
% \label{subsec:repair}

% \mypara{Combining Error Localization and Local Repairs}
\lstDeleteShortInline{|}
