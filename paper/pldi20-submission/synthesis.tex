\section{Repair Synthesis from Templates}
\label{sec:synthesis}
\toolname uses a \emph{enumerative} program synthesis algorithm to fully repair
a program using the predicted templates and locations of our neural network
models. In \autoref{subsec:location-rank}, we show how the predicted error
locations are used if \emph{multiple} of them have to be fixed and in
\autoref{subsec:local-synthesis}, we present our synthesis algorithm for
producing \emph{local repairs} for a given location of the input program.
% Finally, we briefly show in \autoref{subsec:repair} how we use the
% aforementioned techniques to \emph{fully repair} an ill-typed program.

\subsection{Ranking Error Locations}
\label{subsec:location-rank}

\mypara{Error Location Confidence}
Recall from \autoref{sec:templ-pred} that for a given program location, our
$\Model$ generates a confidence score $\Runit$ of that location containing an
error that needs to be fixed. Each location in the type-error slice is assigned
a confidence score $\Runit$ from our $\Model$. Furthermore, for each location we
also generate the repair templates' confidence scores with our $\ModelT$.

Our synthesis algorithm ranks all the locations based on their
confidence scores $\Runit$. For each location, it runs through the list
of templates (also sorted by score) and tries to repair that location using that
template. If the algorithm fails to synthesize a
program that type-checks, the next
location is considered, until it succeeds.

\mypara{Multiple Error Locations}
In practice, however, it is very often that more that one location may need to
be repaired. Therefore, our previous approach will fail to generate a solution
in many cases. So we modify our simple algorithm to synthesize programs with
multiple error locations.

Let the confidence scores $\Runit$ for all locations
in the type error slice from our error localization model $\Model$ be
$(l_1, c_1), \dots, (l_k, c_k)$, where $l_i$ is a location in the type-error
slice and $c_i$ its confidence score. We assume for simplicity
that the probabilities $c_i$ are independent.
With that assumption, the probability that both locations $l_i$ and $l_j$
need to be fixed is $c_i \cdot c_j$. The same holds for more that two locations.

Therefore, we extend our simple algorithm of fixing one location $l$ at a time,
which only considers the confidence scores of single locations, by adding the
combined scores of tuples, triples, etc. of type-error slice locations. In
practice we only consider up to \emph{five} locations to be fixed
simultaneously; any more than that takes too much time to generate and
has too small a chance of leading to an actual good solution.
%% TODO: maybe show how many changes happened to our dataset and how big is type-error slice on average.


\subsection{Local Synthesis from Templates}
\label{subsec:local-synthesis}

\mypara{Enumerative Program Synthesis}
Our synthesis algorithm is a classic \emph{enumerative} program synthesis method
guided by an input template. Enumerative synthesis searches all possible
expressions over a language, starting from an empty program, until a high-level
specification is reached. In our case, we only try to synthesize a part of the
program that already captures the intent of the user and therefore our only
specification is if the repaired program is type-safe. However, we can also
extend this specification by allowing our algorithm to search for programs that
type-check but have the user's desired type signature for the specific program.

Given a location $l$ and a template $t$, our algorithm searches over all
possible expressions over \lang that will satisfy those goals by generating a
\emph{local repair} that expands the $t$'s GAST. One advantage of this technique
is that we can exploit the expression $e$ at location $l$ to further guide our
synthesis, since subexpressions used by the programmer at $l$ are usually reused
for their final repair.

\begin{algorithm}[t]
    \caption{Local Repair Algorithm}
    \label{algo:local-repair-algo}
    \renewcommand{\algorithmicrequire}{\textbf{Input:}}
    \renewcommand{\algorithmicensure}{\textbf{Output:}}
    \begin{algorithmic}[1]
    \Require{Language Grammar \lang, Program $P$, Template $T$, Repair Location $L$, Max Repair Depth $D$}
    \Ensure{Local Repairs $R$}
    \Procedure{Enumerate}{$\lang, P, T, L, D$}
    \State $R \gets \emptyset$
    \ForAll{$d \in [1 \dots D]$}
      \State $\tilde{\alpha} \gets$ \Call{NonTerminalsAt}{$T, d$}
      \ForAll{$\alpha \in$ \Call{RankNonTerminals}{$\tilde{\alpha}, P, L$}}
        \If{\Call{IsHole}{$\tilde{\alpha}$}}
          \State $Q \gets$ \Call{GrammarRules}{$\lang$}
          \State $\tilde{\beta} \gets \{\beta\:|\:(\alpha, \beta) \in Q\}$
          \ForAll{$\beta \in$ \Call{RankRules}{$\tilde{\beta}, T$}}
            \State $\hat{T} \gets$ \Call{ApplyRule}{$T, (\alpha, \beta)$}
          \EndFor
        \Else
          \ForAll{$\sigma \in$ \Call{GetTerminals}{$P, \alpha, \lang$}}
            \State $\hat{T} \gets$ \Call{ReplaceNode}{$T, \alpha, \sigma$}
          \EndFor
        \EndIf
        \State $\hat{P} \gets$ \Call{ReplaceExprAt}{$P, L, \hat{T}$}
        \If{\Call{TypeCheck}{$\hat{P}$}}
          \State $R \gets R \cup \{\hat{P}\}$
        \EndIf
      \EndFor
    \EndFor
    \State \Return{$R$}
    \EndProcedure
    \end{algorithmic}
\end{algorithm}


\mypara{Generating Local Repairs with Templates}
Using our \textsc{Enumerate} method, we lazily generate local repairs $R$ for
each subset of locations that is highest in confidence score. The
\textsc{Enumerate} method starts to fill in a template $T$ for location $L$ of
the program $P$ based on the context-free grammar $\lang$. It starts from the
parent node at the first level of the template $T$ and incrementally moves
down the tree. When a hole is found in the tree, the algorithm tries to expand
the tree one more level using $\lang$'s production rules $R$. The production
rules are considered in an ranked order based on the subexpressions that already
appear in $P$'s location $L$ and the template $T$. It then applies it to the
template $T$. If the node $\tilde{\alpha}$ was not a hole, terminals from the
the program $P$, the expression $\alpha$ at the location $L$ and the grammar
$\lang$ are used to fill that node, depending on what terminals where used from
$\repairLang$. For example, $\repairLang$'s operator $\oplus$ can be replaced
with $+,\:-,\:\etc$

After the template $T$ had some changes applied to it, we get an
\emph{instantiated} template $\tilde{T}$, which then replaces the expression at
location $L$. If the new program $\tilde{P}$ type-checks, it is inserted into
the list of generated solutions $R$. The list of candidate solutions $R$ is
generated lazily in practice and the top-N can be requested from the user
depending on the level of feedback that is needed.


% \subsection{Program Repair}
% \label{subsec:repair}

% \mypara{Combining Error Localization and Local Repairs}
