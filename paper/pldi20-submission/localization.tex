%%%
%%%  FIXME: This entire section is copy-pasted from NATE! Has to change a lot!
%%%

\section{Predicting Error Locations}
\label{sec:localization}

In this section, we briefly describe our approach to localizing type errors, by
introducing our feature set and the models we used. These we also be used for
predicting repair templates with some changes in \autoref{sec:templ-pred}.

Stemming from previous work, we define \lang (\autoref{fig:syntax}), a simple
lambda calculus with integers, booleans, pairs, and lists. For our error
localization algorithm, we aim to instantiate the $\blamesym$ function of
\autoref{fig:api}, which takes as input a $\Model$ of type errors and an
ill-typed program $e$, and returns an ordered list of subexpressions from $e$
paired with the confidence score $\Conf$ that they should be blamed.

A $\Model$ is produced by $\trainsym$, which performs supervised learning on a
training set of feature vectors $\V$ and (boolean) labels $\B$. Once trained, we
can $\evalsym$uate a $\Model$ on a new input, producing the confidence $\Conf$
that the blame label should be applied. We describe multiple $\Model$s and their
instantiations of $\trainsym$ and $\evalsym$ (\autoref{sec:models}).

Of course, the $\Model$ expects feature vectors $\V$ and blame labels $\B$, but
we are given program pairs. So our first step must be to define a suitable
translation from program pairs to feature vectors and labels, \ie we must define
the $\extractsym$ function in \autoref{fig:api}. We model features as
real-valued functions of terms, and extract a feature vector for each
\emph{subterm} of the ill-typed program (\autoref{sec:features}). Then we define
the blame labels for the training set to be the subexpressions that changed
between the ill-typed program and its subsequent fix, and model $\blamesym$ as a
function from a program pair to the set of expressions that changed
(\autoref{sec:labels}). The $\extractsym$ function, then, extracts $\featuresym$
from each subexpression and computes the blamed expressions according to
$\labelsym$.

\begin{figure}
\small
\centering
  \begin{minipage}[c]{\linewidth}
  \[
  \boxed{
  \begin{array}{rcl}
  e & ::=    & x \spmid \efun{x}{e} \spmid \eapp{e}{e} \spmid \elet{x}{e}{e} \\
    & \spmid & n \spmid \eplus{e}{e}\\
    & \spmid & b \spmid \eif{e}{e}{e} \\
    & \spmid & \epair{e}{e} \spmid \epcase{e}{x}{x}{e} \\
    & \spmid & \enil \spmid \econs{e}{e} \spmid \ecase{e}{e}{x}{x}{e} \\[0.05in]

  n & ::= &  0, 1, -1, \ldots \\[0.05in]

  b & ::= &  \etrue \spmid \efalse \\[0.05in]

  t & ::= & \alpha \spmid \tbool \spmid \tint \spmid \tfun{t}{t} \spmid \tprod{t}{t} \spmid \tlist{t} \\[0.05in]
  \end{array}
  }
  \]
  \captionof{figure}{Syntax of \lang}
  \label{fig:ml-syntax}
  \end{minipage}
  \begin{minipage}[c]{\linewidth}
    \[
    \boxed{
    \begin{array}{rcl}
    e & ::=    & x \spmid \_  \spmid \efun{x}{e} \spmid \eapp{e}{e} \spmid \elet{x}{e}{e} \\
      & \spmid & n \spmid \ebop{e}{e} \spmid \eif{e}{e}{e} \\
      & \spmid & \epair{e}{e} \spmid \epcase{e}{x}{x}{e} \\
      & \spmid & \enil \spmid \econs{e}{e} \spmid \ecase{e}{e}{x}{x}{e} \\[0.05in]
    \end{array}
    }
    \]
    \captionof{figure}{Syntax of \repairLang}
    \label{fig:rtl-syntax}
  \end{minipage}
\end{figure}



\subsection{Features}
\label{sec:features}
The first issue we must tackle is formulating our learning task in machine
learning terms. We are given programs over \lang, but learning algorithms expect
to work with \emph{feature vectors} $\V$ --- vectors of real numbers, where each
column describes a particular aspect of the input. Thus, our first task is to
convert programs to feature vectors.

We choose to model a program as a \emph{set} of feature vectors, where each
element corresponds an expression in the program. Thus, given the |sumList|
program in \autoref{fig:sumList} we would first split it into its constituent
subexpressions and then transform each subexpression into a single feature
vector. We group the features into five categories, using \autoref{tab:sumList}
as a running example of the feature extraction process.

\paragraph{Local syntactic features}
These features describe the syntactic category of each expression $e$. In other
words, for each production of $e$ in \autoref{fig:syntax} we introduce a feature
that is enabled (set to $1$) if the expression was built with that production,
and disabled (set to $0$) otherwise. For example, the \IsNil feature in
\autoref{tab:sumList} describes whether an expression is the empty list $\enil$.

We distinguish between matching on a list vs.\ on a pair, as this affects the
typing derivation. We also assume that all pattern matches are well-formed ---
\ie all patterns must match on the same type. Ill-formed match expressions would
lead to a type error; however, they are already effectively localized to the
match expression itself. We note that this is not a \emph{fundamental}
limitation, and one could easily add features that specify whether a match
\emph{contains} a particular pattern, and thus have a match expression that
enables multiple features.

\paragraph{Contextual syntactic features}
These are similar to local syntactic features, but lifted to describe the parent
and children of an expression. For example, the \IsCaseListP feature in
\autoref{tab:sumList} describes whether an expression's \emph{parent} matches on
a list. If a particular $e$ does not have children (\eg a variable $x$) or a
parent (\ie the root expression), we leave the corresponding features disabled.
This gives us a notion of the \emph{context} in which an expression occurs,
similar to the \emph{n-grams} used in linguistic models
\citep{Hindle2012-hf,Gabel2010-el}.

\paragraph{Expression size}
We also propose a feature representing the \emph{size} of each expression, \ie
how many subexpressions does it contain? For example, the \ExprSize feature in
\autoref{tab:sumList} is set to three for the expression |sumList tl| as it
contains three expressions: the two variables and the application itself. This
allows the model to learn that, \eg, expressions closer to the leaves are more
likely to be blamed than expressions closer to the root.

\paragraph{Typing features}
A natural way of summarizing the context in which an expression occurs is with
\emph{types}. Of course, the programs we are given are \emph{untypeable}, but we
can still extract a \emph{partial} typing derivation from the type checker and
use it to provide more information to the model.

A difficulty that arises here is that, due to the parametric type constructors
$\tfun{\cdot}{\cdot}$, $\tprod{\cdot}{\cdot}$, and $\tlist{\cdot}$, there is an
\emph{infinite} set of possible types --- but we must have a \emph{finite} set
of features. Thus, we abstract the type of an expression to the set of type
constructors it \emph{mentions}, and add features for each type constructor that
describe whether a given type mentions the type constructor. For example, the
type $\tint$ would only enable the $\tint$ feature, while the type
$\tfun{\tint}{\tbool}$ would enable the $\tfun{\cdot}{\cdot}$, $\tint$, and
$\tbool$ features.

We add these features for parent and child expressions to summarize the context,
but also for the current expression, as the type of an expression is not always
clear \emph{syntactically}. For example, the expressions |tl| and |sumList tl|
in \autoref{tab:sumList} both enable \HasTypeList, as they are both inferred to
have a type that mentions $\tlist{\cdot}$.
%constructor.

Note that our use of typing features in an ill-typed program subjects us to
\emph{traversal bias} \citep{McAdam1998-ub}. For example, the |sumList tl|
expression might alternatively be assigned the type $\tint$. Our models will
have to learn good localizations in spite this bias (see
\autoref{sec:evaluation}).

\paragraph{Type error slice}
Finally, we wish to distinguish between changes that could fix the error, and
changes that \emph{cannot possibly} fix the error. Thus, we compute a minimal
type error \emph{slice} for the program (\ie the set of expressions that
contribute to the error), and add a feature that is enabled for expressions that
are part of the slice. The \InSlice feature in \autoref{tab:sumList} indicates
whether an expression is part of such a minimal slice, and is enabled for all of
the sampled expressions except for |tl|, which does not affect the type error.
If the program contains multiple type errors, we compute a minimal slice for
each error.

In practice, we have found that \InSlice is a particularly important feature,
and thus include a post-processing step that discards all expressions where it
is disabled. As a result, the |tl| expression would never actually be shown to
the classifier. We will demonstrate the importance of \InSlice empirically in
\autoref{sec:feature-utility}.


\subsection{Labels}
\label{sec:labels}
Recall that we make predictions in two stages. First, we use $\evalsym$ to
predict for each subexpression whether it should be blamed, and extract a
confidence score $\Conf$ from the $\Model$. Thus, we define the output of the
$\Model$ to be a boolean label, where ``false'' means the expression
\emph{should not} change and ``true'' means the expression \emph{should} change.
This allows us to predict whether any individual expression should change, but
we would actually like to predict the \emph{most likely} expressions to change.
Second, we \emph{rank} each subexpression by the confidence $\Conf$ that it
should be blamed, and return to the user the top $k$ most likely blame
assignments (in practice $k=3$).


We identify the fixes for each ill-typed program with an expression-level
diff~\citep{Lempsink2009-xf}. We consider two sources of changes. First, if an
expression has been removed wholesale, \eg if $\eapp{f}{x}$ is rewritten to
$\eapp{g}{x}$, we will mark the expression $f$ as changed, as it has been
replaced by $g$. Second, if a new expression has been inserted \emph{around} an
existing expression, \eg if $\eapp{f}{x}$ is rewritten to
$\eplus{\eapp{f}{x}}{1}$, we will mark the application expression $\eapp{f}{x}$
(but not $f$ or $x$) as changed, as the $+$ operator now occupies the original
location of the application.


\subsection{Learning Algorithm}
\label{sec:models}
\lstDeleteShortInline{|} % sigh...

Recall that we formulate type error detection at a single expression as a
supervised classification problem. This means that we are given a training data
set
%$S = \{ \langle v_i, b_i \rangle \}_{i=1}^n$
$S : \List{\V \times \B}$ of labeled examples, and our goal is to use it to
build a \emph{classifier}, \ie a rule that can predict a label $b$ for an input
$v$. Since we apply the classifier on each expression in the program to
determine those that are the most likely to be type errors, we also require the
classifier to output a \emph{confidence score} that measures how sure the
classifier is about its prediction.


There are many learning algorithms to choose from, existing on a spectrum that
balances expressiveness with ease of training (and of interpreting the learned
model). In this section we consider a standard learning algorithm: \emph{neural
networks}. A thorough introduction to these techniques can be found in
introductory machine learning textbooks \citep[\eg][]{Hastie2009-bn}.

Below we briefly introduce each technique by describing the rules it learns, and
summarize its advantages and disadvantages. For our application, we are
particularly interested in three properties -- expressiveness, interpretability
and ease of generalization. Expressiveness measures how complex prediction rules
are allowed to be, and interpretability measures how easy it is to explain the
cause of prediction to a human. Finally ease of generalization measures how
easily the rule generalizes to examples that are not in the training set; a rule
that is not easily generalizable might perform poorly on an unseen test set even
when its training performance is high.


\paragraph{Neural Networks}
The last (and most complex) model we use is a type of neural network called a
\emph{multi-layer perceptron} (see \citealt{Nielsen2015-pu} for an introduction
to neural networks). A multi-layer perceptron can be represented as a directed
acyclic graph whose nodes are arranged in layers that are fully connected by
weighted edges. The first layer corresponds to the input features, and the final
to the output.
%The output of a node $v$ in an internal layer is given by:
The output of an internal node $v$ is
\[ h_v = g\,(\sum_{j \in N(v)}\!W_{jv} h_j ) \] where $N(v)$ is the set of nodes
in the previous layer that are adjacent to $v$, $W_{jv}$ is the weight of the
$(j, v)$ edge and $h_j$ is the output of node $j$ in the previous layer. Finally
$g$ is a non-linear function, called the activation function, which in recent
work is commonly chosen to be the \emph{rectified linear unit} (ReLU), defined
as $g(x) = \mathsf{max}(0,x)$ \citep{Nair2010-xg}. The number of layers, the
number of neurons per layer, and the connections between layers constitute the
\emph{architecture} of a neural network. In this work, we use relatively simple
neural networks which have an input layer, a single hidden layer and an output
layer.

A major advantage of neural networks is their ability to discover interesting
combinations of features through non-linearity, which significantly reduces the
need for manual feature engineering, and allows high expressivity. On the other
hand, this makes the networks particularly difficult to interpret and also
difficult to generalize unless vast amounts of training data are available.

\lstMakeShortInline[mathescape=true]{|}
