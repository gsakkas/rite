\section{Related Work}
\label{sec:related-work}

There is a vast literature on automatically repairing or patching programs:
% for specific properties (\eg null-dereferences, memory safety),
we focus on the most closely related work on providing feedback for novice errors.

\mypara{Example-Based Feedback}
%
Recent work has looked at providing \emph{counterexamples} that show how a
program went wrong, for type errors \cite{Seidel2016-ul} or for general
correctness properties where the generated inputs show divergence from a
reference implementation or other correctness oracle~\cite{Song_2019}. In
contrast, we provide feedback on how to fix the error.

\mypara{Fault Localization} Several authors have studied the problem of
\emph{fault localization}, \ie winnowing down the set of locations that are
relevant for the error, often using slicing
\citep{Wand1986-nw,Haack2003-vc,Tip2001-qp,Rahli2015-tt}, counterfactual typing
\citep{Chen2014-gd} or bayesian methods \citep{Zhang2014-lv}.
%
\textsc{Nate}~\citep{Seidel:2017} introduced the BOAT representation,
and showed it could be used for accurate localization.
%
We aim to go beyond localization, into suggesting concrete \emph{changes} that
novices can make to understand and fix the problem.

\mypara{Repair-model based feedback}
%
\textsc{Seminal} \citep{Lerner2007-dt} \emph{enumerates} minimal fixes using an
expert-guided heuristic search.
%
The above approach is generalized to general correctness properties by
\cite{singh2013} which additionally performs a \emph{symbolic} search using a
set of expert provided \emph{sketches} that represent possible repairs.
%
In contrast, \toolname learns a template of repairs from a corpus yielding
higher quality feedback (\S~\ref{sec:eval}).

\mypara{Corpus-based feedback}
%
\textsc{Clara} \citep{Gulwani_2018} uses code and execution traces to match a
given incorrect program with a ``nearby'' correct solution obtained by
clustering all the correct answers for a particular task. The matched
representative is used to extract repair expressions.
%
Similarly, \textsc{Sarfgen} \citep{Wang_2018} focuses on structural and
control-flow similarity of programs to produce repairs, by using AST vector
embeddings to calculate distance metrics (to ``nearby'' correct
programs) more robustly.
%
\textsc{Clara} and \textsc{Sarfgen} are data-driven, but both assume
there is a ``close'' correct sample in the corpus.
%
In contrast, \toolname has a more general philosophy that \emph{similar errors
have similar repairs}: we extract generic fix templates that can be applied to
arbitrary programs whose errors (BOAT vectors) are similar.
%
The \textsc{Tracer} system \citep{TRACER2018} is closest in philosophy to ours,
except that it focuses on single-line compilation errors for C programs, where
it shows that NLP-based methods like sequence-to-sequence predicting DNNs can
effectively suggest repairs, %\eg \verb+scanf("%d", a)+ should be converted to
%\verb+scanf("%d", %a)+
but this does not scale up to fixing general type errors.
%
We have found that \ocaml's relatively simple
\emph{syntactic} structure but rich \emph{type}
structure make token-level seq-to-seq methods quite imprecise (\eg
\emph{deleting} offending statements suffices to ``repair'' C but yields
ill-typed \ocaml) necessitating \toolname's higher-level semantic features and
(learned) repair templates.

\textsc{Hoppity} \citep{Dinella_2020} is a \dnn-based approach for fixing buggy
Javascript programs. \textsc{Hoppity} considers programs as graphs that are
feeded to a \textit{Graph Neural Network} to produce fixed-length embeddings,
that are in turn used in a LSTM model that generates a sequence of primitive
edits of the program graph. To our knowledge, \textsc{Hoppity} is one of the few
pieces of art that can address multiple error locations. This method solely
relies on the underlying learned models to generate a sequence of edits.
\toolname, however, uses the learned models to get appropriate error locations
and fix templates, but then uses a synthesis procedure to always generate a
number of type-correct programs. \textsc{Hoppity} on the other side, doesn't
guarantee returning valid Javascript programs.

\textsc{Getafix} \citep{Bader_2019} and \textsc{Revisar} \citep{Rolim_2018} are
two more systems that learn fix patterns using AST-level differencing on a
corpus of past bug fixes. They both use \textit{anti-unification}
\citep{Kutsia_2014} for generalizing expressions and, thus, grouping together
fix patterns. They cluster together bug fixes in order to reduce the search
space of candidate patches. While \textsc{Revisar} \citep{Rolim_2018} ends up
with one fix pattern per bug category using anti-unification, \textsc{Getafix}
\citep{Bader_2019} builds a hierarchy of patterns that also include the context
of the edit to be made. They both keep before and after expression pairs as
their fix patterns, and they use the before expression as a means to match an
expression in a new buggy program and replace it with the after expression.
While these methods are quite effective, they are only applicable in recurring
bug categories e.g. how to deal with a null pointer exception. \toolname on the
other hand, attempts to generalize fix patterns even more by using the GAST
abstractions, and predicts proper error locations and fix patterns with a
learned model from the corpus of bug fixes, so they can be applied into a
diverse variety of bugs.

\textsc{Prophet} \citep{Long_2016} is another technique that uses a corpus of
fixed buggy programs to learn a probabilistic model that will rank candidate
patches. Patches are generated using a set of predefined transformation schemas
and condition synthesis. \textsc{Prophet} uses logistic regression to learn the
parameters of this model and uses over 3500 extracted program features to do so.
It also uses an instrumented recompile of a faulty program together with some
failing input test cases to identify what program locations are of interest.
While this method can be highly accurate for error localization, their
experimental results show that it can take up to 2 hours to produce a valid
candidate fix. In contrast, \toolname's pretrained models make finding proper
error locations and possible fix templates more robust.

