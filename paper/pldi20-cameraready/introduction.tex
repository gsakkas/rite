\section{Introduction}
\label{sec:intro}

% The problem
%
Languages with Hindley-Milner style, unification-based inference offer the
benefits of static typing with minimal annotation overhead. The catch, however,
is that programmers must first ascend the steep learning curve associated with
understanding the \emph{error messages} produced by the compiler.
%
While \emph{experts} can, usually, readily decipher the errors, and view them as
invaluable aids to program development and refactoring, \emph{novices} are
typically left quite befuddled and frustrated, without a clear idea of
\emph{what} the problem is~\citep{Wand1986-nw}.
%
Owing to the importance of the problem, several authors have proposed methods to
help debug type errors, typically, by \emph{slicing} down the program to the
problematic locations~\citep{Haack2003-vc, Rahli2015-tt}, by \emph{enumerating}
possible causes \citep{Lerner2007-dt, Chen2014-gd}, or by \emph{ranking} the
possible locations using MAX-SAT \citep{Pavlinovic2014-mr},
Bayesian~\citep{Zhang2014-lv} or statistical analysis~\citep{Seidel:2017}.
%
While valuable, these approaches at best help localize the problem but students
are still left in the dark about how to \emph{fix} their code.

% Repairs as error messages?
\mypara{Repairs as Feedback}
%
Several recent papers have proposed an inspiring new line of
attack on the feedback problem: using techniques from synthesis
to provide feedback in the form of \emph{repairs} that students
can apply to improve their code.

One way these repairs are found is by symbolically searching a space of candidate
programs circumscribed by an expert-defined repair model
\citep{singh2013,HeadGSSFDH17}.
%
However, for type errors, the space of candidate repairs is massive.
It is quite unclear whether a small set of repair models \emph{exists}
or even if it does, what it \emph{looks like}. More importantly,
to scale, it is essential that we remove the requirement that an
expert carefully curate some set of candidate repairs.

Another way to generate repairs is via the observation that \emph{similar
programs} have similar repairs, \ie by calculating ``diffs'' from the student's
solution to the ``closest'' \emph{correct} program
~\citep{Gulwani_2018,Wang_2018}.
%
However, this approach requires a corpus of similar programs,
whose syntax trees or execution traces can be used to match
each incorrect program with a ``correct'' version that is
used to provide feedback. Programs with static type errors
have no execution traces.
%
More importantly, we desire a means to generate feedback
for \emph{new} programs that novices write, and hence
cannot rely on matching against some (existing) correct
program.

\mypara{Analytic Program Repair}
%
In this work, we present a novel error repair strategy called \emph{Analytic
Program Repair} that uses supervised learning instead of manually crafted repair
models or matching against a corpus of correct code.
%
Our strategy is based on the key insight that \emph{similar errors} have similar
repairs and realizes this insight by using a training dataset of pairs of
ill-typed programs and their fixed versions to:
%
(1)~\emph{learn} a collection of candidate repair templates
    by abstracting and partitioning the edits made in the
    training set into a representative set of templates;
%
(2)~\emph{predict} the appropriate template from a given error,
    by training multi-class classifiers on the repair templates
    used in the training set;
%
(3)~\emph{synthesize} a concrete repair from the template
   by enumerating and ranking correct (\eg well-typed)
   terms matching the predicted template,
%
thereby, generating a fix for a candidate program.
%
Critically, we show how to perform the crucial abstraction
from a particular \emph{program} to an abstract \emph{error}
by representing programs via \emph{bag-of-abstracted-terms} (BOAT)
\ie as numeric vectors of syntactic and semantic features \citep{Seidel2017-ko}.
%
This abstraction lets us train predictors over high-level
code features, \ie to learn correlations between features
that cause errors and their corresponding repairs, allowing
the analytic approach to generalize beyond matching against
existing programs.

\mypara{\toolname}
%
We have implemented our approach in \toolname: a type error reporting
tool for \ocaml programs. We train (and evaluate) \toolname on a set of
over 4,500 ill-typed \ocaml programs drawn from two years of an
introductory programming course.
%
Given a new ill-typed program, \toolname generates a list of potential
solutions ranked by likelihood and an \emph{edit-distance} metric.
We train \toolname on programs from one year and evaluate in several
ways.
%
First, we measure its \emph{accuracy}: we show that \toolname correctly predicts
the right repair template 69\% of the time when considering the top three
templates and surpasses 80\% when we consider the top six.
%
Second, we measure its \emph{efficiency}: we show that \toolname is able to
synthesize a concrete repair within 20 seconds 70\% of the time.
%
Finally, we measure the \emph{quality} of the generated messages via a user
study with 29 participants and show that humans perceive both \toolname's edit
locations and final repair quality to be better than those produced by \seminal,
a state-of-the-art OCaml repair tool \citep{Lerner2007-dt} in a
statistically-significant manner.


% First, what is a repair and when are two repairs similar?
% Second, what is an error and when are two errors similar?
% Third, how do we automatically match (similar) errors to (similar) repairs?
%
% \mypara{Repairs}
% \mypara{Errors}
% \mypara{Predicting Repairs from Errors}
% \tool converts this insight into an error reporting algorithm by addressing three crucial questions.
%%* How to \emph{curate} a collection of generic repairs?
%%* How to \emph{predict} which generic repair applies to a given error?
%%* How to \emph{synthesize} a concrete repair that fixes the given error?
%%
%%
%%We must address two crucial quie
%%To make this insight actionable, we
%%To convert this insight into an er
%%That is, rather than associating errors and repo
%%
%%
%%generalizes the key observation
%%
%%that is inspired by the synthesis approaches, but generalizes their
%%
%%Inspired by the above,
%%
%%
%%% Our approach
%%- RITE:
    %%1. repairs? learn family of repair-templates from student dataset (generalizable, fine-grained edits, not whole program)
    %%2. similar? train multi-labeled classifiers on BOAT; given new program predict location+template
    %%3. message? enumerative synthesis to fill in template at location using type-checker as oracle.
