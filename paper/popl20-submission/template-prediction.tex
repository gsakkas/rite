\section{Predicting Repair Templates}
\label{sec:templ-pred}

In this section, we extend our API from \autoref{sec:localization}, so we are
able to predict repair templates $\T$ for a given location of the program. Our
goal is to define the $\evalsym$ function in \autoref{fig:api}, in terms of the
simple language \repairLang (\autoref{fig:syntax}), which takes a $\ModelT$ and
a feature vector $\V$ of a specific subexpression as an input and produces a
confidence score $\C$ for each of the chosen templates $\T$. Any given template
$\T$ is an expression $e$ of the \repairLang, which a simple lambda calculus
with integers, booleans, pairs, and lists.

Firstly, a $\ModelT$ is produced by $\trainTsym$, which performs supervised
learning on a training set of feature vectors $\V$ and a fixed-length of
(boolean) labels $\B$, that each represent a template $\T$. In our case, only
one template can be applied to a given location, so we define here our problem
as a \emph{multi-class classification} problem. Once trained, we can make
predictions on new inputs, producing template confidences $\Runit$ for each
template $\T$.

Similarly to \autoref{sec:localization}, our $\ModelT$s expect feature vectors
$\V$ and boolean labels $\B$, both of a fixed length for each specific location.
Therefore, we define similarly to $\extractsym$, the function $\extractTsym$ in
\autoref{fig:api}. We use again $\diffsym$ to get the set of changed expressions
of a given program pair. Those are then used by the function $\clustersym$ to
get the repair templates by grouping different expressions together based on
some similarity metric and thus reducing their number and making them more
concrete. The $\extractTsym$ function, then, extracts $\featuresym$ from each
subexpression, acquired by $\diffsym$ but limited to the type-error slice (TODO:
ref) and assigns the boolean labels based on the templates $\T$ according to
$\clustersym$, with only one being $\etrue$ at a time


\subsection{RTL: Repair Template Language}
\label{subsec:lang}

\mypara{Syntax of RTL}
\begin{figure}
\small
\centering
\begin{minipage}[c]{\linewidth}
  \[
  \boxed{
  \begin{array}{rcl}
  e & ::=    & x \spmid \efun{x}{e} \spmid \eapp{e}{e} \spmid \elet{x}{e}{e} \\
    & \spmid & n \spmid \eplus{e}{e}\\
    & \spmid & b \spmid \eif{e}{e}{e} \\
    & \spmid & \epair{e}{e} \spmid \epcase{e}{x}{x}{e} \\
    & \spmid & \enil \spmid \econs{e}{e} \spmid \ecase{e}{e}{x}{x}{e} \\[0.05in]

  n & ::= &  0, 1, -1, \ldots \\[0.05in]

  b & ::= &  \etrue \spmid \efalse \\[0.05in]

  t & ::= & \alpha \spmid \tbool \spmid \tint \spmid \tfun{t}{t} \spmid \tprod{t}{t} \spmid \tlist{t} \\[0.05in]
  \end{array}
  }
  \]
  \captionof{figure}{Syntax of \repairLang}
  \label{fig:syntax}
\end{minipage}
\begin{minipage}[c]{\linewidth}
  \lstDeleteShortInline{|} % sigh...
  \[
  \boxed{
  \begin{array}{lcl}
    \V           & \defeq & \List{\R}\\
    \Runit       & \defeq & \{r \in \R\ |\ 0 \le r \le 1\} \\ % [0,1]\\
    \T           & \defeq & e\\
    \featuresym  & : & \List{e \to \R} \\
    \labelsym    & : & e \times e \to \List{e} \\
    \extractTsym & : & \featuresym \to e \times e \to \List{\V \times \List{\T}} \\
    \trainTsym   & : & \List{\V \times \List{\T}} \to \Model \\
    \predictTsym & : & \Model \to \V \to \List{\Runit} \\
    \midrule
    \repairsym   & : & \Model \to e \to \List{e \times \Runit \times \List{\Runit}}
  \end{array}
  }
  \]
  \lstMakeShortInline{|}
  \captionof{figure}{
    A high-level API for converting program pairs to
    feature vectors and template labels.
  }
  \label{fig:api}
\end{minipage}
\end{figure}



\mypara{Extracting Templates from Dataset Repairs}

\mypara{Example}


\subsection{Clustering the Templates}
\label{subsec:clustering}

\mypara{Creating GASTs from RTL Templates} (Generic ASTs.)

\mypara{GAST Similarity Metric}

\mypara{Clustering}

\mypara{Example}


\subsection{Multi-class Classification}
\label{subsec:multi-class}

\mypara{Assigning Templates as Labels}

\mypara{Multi-class \dnn{}s}
