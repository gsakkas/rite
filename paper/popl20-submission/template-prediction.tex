\section{Predicting Repair Templates}
\label{sec:templ-pred}

In this section, we extend our API from \autoref{sec:localization}, so we are
able to predict repair templates $\T$ for a given location of the program. Our
goal is to define the $\evalsym$ function in \autoref{fig:api}, in terms of the
simple language \repairLang (\autoref{fig:syntax}), which takes a $\ModelT$ and
a feature vector $\V$ of a specific subexpression as an input and produces a
confidence score $\C$ for each of the chosen templates $\T$. Any given template
$\T$ is an expression $e$ of the \repairLang, which a simple lambda calculus
with integers, booleans, pairs, and lists.

Firstly, a $\ModelT$ is produced by $\trainTsym$, which performs supervised
learning on a training set of feature vectors $\V$ and a fixed-length of
(boolean) labels $\B$, that each represent a template $\T$. In our case, only
one template can be applied to a given location, so we define here our problem
as a \emph{multi-class classification} problem. Once trained, we can make
predictions on new inputs, producing template confidences $\Runit$ for each
template $\T$.

Similarly to \autoref{sec:localization}, our $\ModelT$s expect feature vectors
$\V$ and boolean labels $\B$, both of a fixed length for each specific location.
Therefore, we define similarly to $\extractsym$, the function $\extractTsym$ in
\autoref{fig:api}. We use again $\diffsym$ to get the set of changed expressions
of a given program pair. Those are then used by the function $\clustersym$ to
get the repair templates by grouping different expressions together based on
some similarity metric and thus reducing their number and making them more
concrete. The $\extractTsym$ function, then, extracts $\featuresym$ from each
subexpression, acquired by $\diffsym$ but limited to the type-error slice (TODO:
ref) and assigns the boolean labels based on the templates $\T$ according to
$\clustersym$, with only one being $\etrue$ at a time


\subsection{RTL: Repair Template Language}
\label{subsec:lang}

In this section, we define the language we are going to use to represent
templates and how we can acquire those templates from our dataset.

\mypara{Syntax of RTL}
We define here our Repair Template Language (RTL) as a simple lambda calculus
(\repairLang in \autoref{fig:syntax}) with integers, booleans, pairs, and lists.
This language will be used to abstract and define repair templates $\T$. Our RTL
contains all the expressions that \lang included, but with four significant
changes.

\begin{enumerate}
    \item Variable names $x$ used for functions, variables and patterns are left
    \emph{unspecified}. That means that $x$ denotes that that place of the code
    has a variable name, but it is unknown at that point.
    \item Literal values $n$ can be any integer or floating point number,
    boolean values or character and string. Same as variables, these are left
    unspecified.
    \item Operators $\oplus$ are also left unspecified.
    \item A \emph{wildcard} expression $\_$ is added, which is used to denote
    that \emph{any} possible expression from \repairLang can be used to replace
    it.
\end{enumerate}

\begin{figure}
\small
\centering
  \begin{minipage}[c]{\linewidth}
  \[
  \boxed{
  \begin{array}{rcl}
  e & ::=    & x \spmid \efun{x}{e} \spmid \eapp{e}{\bar{e}} \spmid \elet{x}{e}{e} \\
    & \spmid & n \spmid b \spmid \eplus{e}{e} \spmid \eif{e}{e}{e} \\
    & \spmid & \epair{e}{e} \spmid \epcase{e}{x}{x}{e} \\
    & \spmid & \enil \spmid \econs{e}{e} \spmid \ecase{e}{e}{x}{x}{e} \\[0.05in]

  n & ::= &  0, 1, -1, \ldots \\[0.05in]

  b & ::= &  \etrue \spmid \efalse \\[0.05in]

  t & ::= & \alpha \spmid \tbool \spmid \tint \spmid \tfun{t}{t} \spmid \tprod{t}{t} \spmid \tlist{t} \\[0.05in]
  \end{array}
  }
  \]
  \captionof{figure}{Syntax of \lang}
  \label{fig:ml-syntax}
  \end{minipage}
  \begin{minipage}[c]{\linewidth}
    \[
    \boxed{
    \begin{array}{rcl}
    e & ::=    & \_  \spmid \hat{x} \spmid \efun{\hat{x}}{e} \spmid \eapp{\hat{x}}{\bar{e}} \spmid \elet{\hat{x}}{e}{e} \\
      & \spmid & \hat{n} \spmid \ebop{e}{e} \spmid \eif{e}{e}{e} \\
      & \spmid & \epair{e}{e} \spmid \epcase{e}{\hat{x}}{\hat{x}}{e} \\
      & \spmid & \enil \spmid \econs{e}{e} \spmid \ecase{e}{e}{\hat{x}}{\hat{x}}{e} \\[0.05in]
    \end{array}
    }
    \]
    \captionof{figure}{Syntax of \repairLang}
    \label{fig:rtl-syntax}
  \end{minipage}
\end{figure}


\mypara{Generic ASTs (GASTs)}
Using our RTL, we can now define \emph{Generic ASTs}. GASTs are abstract syntax
trees that represent expressions that can have unspecified details, like
variable names or numerical values. Therefore, GASTs can represent templates
$\T$, defined as expressions in \repairLang. GASTs are used here to capture the
high-level \emph{structure} of an expression, as a more appropriate means to
define templates. GASTs are extracted from the original ASTs representing
programs in \lang, by removing all the necessary information so \repairLang can
be used. Then these trees can be further \emph{pruned} at a depth $d$ to keep
only the higher level expressions and formulate the templates.


\mypara{Extracting Templates from Dataset Repairs}
Templates $\T$ are extracted by the function $\templatesym$ from our program
pair dataset. Using the $\diffsym$ function first, we acquire the changed
expressions of the dataset. These changes are potentially the templates we are
going to use. However, they contain a lot of \emph{local} information to the
specific program that they were extracted from. So $\templatesym$ takes the
changed expressions defined in \lang and \emph{transforms} them into expressions
in \repairLang, essentially making them into templates $\T$. Those templates are
represented as GASTs and are pruned at a pre-defined depth $d$. The value of $d$
can be as low as 1, but to capture more structure to our templates, slightly
bigger values would work better. We then see in \autoref{subsec:clustering} how we can group together all these templates.

\mypara{Example}
TODO: give an example

\subsection{Clustering the Templates}
\label{subsec:clustering}

\mypara{GAST Similarity Metric}

\mypara{Clustering}

\mypara{Example}


\subsection{Multi-class Classification}
\label{subsec:multi-class}

\mypara{Assigning Templates as Labels}

\mypara{Multi-class \dnn{}s}
