\section{Predicting Repair Templates}
\label{sec:templ-pred}

In this section, we extend our API from \autoref{sec:localization}, so we are
able to predict repair templates $\T$ for a given location of the program. Our
goal is to define the $\evalsym$ function in \autoref{fig:api}, in terms of the
simple language \repairLang (\autoref{fig:syntax}), which takes a $\Model$ and a
feature vector $\V$ of a specific subexpression as an input and produces a
confidence score $\C$ for each of the chosen templates $\T$. Any given template
$\T$ is an expression $e$ of the \repairLang, which a simple lambda calculus
with integers, booleans, pairs, and lists.

Fistly, a $\Model$ is produced by $\trainTsym$, which performs supervised
learning on a training set of feature vectors $\V$ and a fixed-length of
(boolean) labels $\B$, that each represent a template $\T$. In our case, only
one template can be applied to a given location, so we define here our problem
as a \emph{multi-class classification} problem. Once trained, we can make
predictions on new inputs, producing template confidences $\Runit$ for each
template $\T$.

Similarly to \autoref{sec:localization}, our $\Model$s expect feature vectors $\V$ and boolean labels $\B$, both of a fixed length for each specific location. Therefore, we define similarly to $\extractsym$, the function $\extractTsym$ in \autoref{fig:api}.

Of course, the $\Model$ expects feature vectors $\V$ and blame labels $\B$, but
we are given program pairs. So our first step must be to define a suitable
translation from program pairs to feature vectors and labels, \ie we must define
the $\extractsym$ function in \autoref{fig:api}. We model features as
real-valued functions of terms, and extract a feature vector for each
\emph{subterm} of the ill-typed program (\autoref{sec:features}). Then we define
the blame labels for the training set to be the subexpressions that changed
between the ill-typed program and its subsequent fix, and model $\blamesym$ as a
function from a program pair to the set of expressions that changed
(\autoref{sec:labels}). The $\extractsym$ function, then, extracts $\featuresym$
from each subexpression and computes the blamed expressions according to
$\labelsym$.

\subsection{RTL: Repair Template Language}
\label{subsec:lang}

\mypara{Syntax of RTL}
\begin{figure}
\small
\centering
  \begin{minipage}[c]{\linewidth}
  \[
  \boxed{
  \begin{array}{rcl}
  e & ::=    & x \spmid \efun{x}{e} \spmid \eapp{e}{\bar{e}} \spmid \elet{x}{e}{e} \\
    & \spmid & n \spmid b \spmid \eplus{e}{e} \spmid \eif{e}{e}{e} \\
    & \spmid & \epair{e}{e} \spmid \epcase{e}{x}{x}{e} \\
    & \spmid & \enil \spmid \econs{e}{e} \spmid \ecase{e}{e}{x}{x}{e} \\[0.05in]

  n & ::= &  0, 1, -1, \ldots \\[0.05in]

  b & ::= &  \etrue \spmid \efalse \\[0.05in]

  t & ::= & \alpha \spmid \tbool \spmid \tint \spmid \tfun{t}{t} \spmid \tprod{t}{t} \spmid \tlist{t} \\[0.05in]
  \end{array}
  }
  \]
  \captionof{figure}{Syntax of \lang}
  \label{fig:ml-syntax}
  \end{minipage}
  \begin{minipage}[c]{\linewidth}
    \[
    \boxed{
    \begin{array}{rcl}
    e & ::=    & \_  \spmid \hat{x} \spmid \efun{\hat{x}}{e} \spmid \eapp{\hat{x}}{\bar{e}} \spmid \elet{\hat{x}}{e}{e} \\
      & \spmid & \hat{n} \spmid \ebop{e}{e} \spmid \eif{e}{e}{e} \\
      & \spmid & \epair{e}{e} \spmid \epcase{e}{\hat{x}}{\hat{x}}{e} \\
      & \spmid & \enil \spmid \econs{e}{e} \spmid \ecase{e}{e}{\hat{x}}{\hat{x}}{e} \\[0.05in]
    \end{array}
    }
    \]
    \captionof{figure}{Syntax of \repairLang}
    \label{fig:rtl-syntax}
  \end{minipage}
\end{figure}



\mypara{Extracting Templates from Dataset Repairs}

\mypara{Example}


\subsection{Clustering the Templates}
\label{subsec:clustering}

\mypara{Creating GASTs from RTL Templates} (Generic ASTs.)

\mypara{GAST Similarity Metric}

\mypara{Clustering}

\mypara{Example}


\subsection{Multi-class Classification}
\label{subsec:multi-class}

\mypara{Assigning Templates as Labels}

\mypara{Multi-class \dnn{}s}
