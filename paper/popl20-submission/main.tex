%% For double-blind review submission, w/o CCS and ACM Reference (max submission space)
\documentclass[acmsmall,review,anonymous]{acmart}\settopmatter{printfolios=true,printccs=false,printacmref=false}
%% For double-blind review submission, w/ CCS and ACM Reference
%\documentclass[acmsmall,review,anonymous]{acmart}\settopmatter{printfolios=true}
%% For single-blind review submission, w/o CCS and ACM Reference (max submission space)
%\documentclass[acmsmall,review]{acmart}\settopmatter{printfolios=true,printccs=false,printacmref=false}
%% For single-blind review submission, w/ CCS and ACM Reference
%\documentclass[acmsmall,review]{acmart}\settopmatter{printfolios=true}
%% For final camera-ready submission, w/ required CCS and ACM Reference
%\documentclass[acmsmall]{acmart}\settopmatter{}


%% Journal information
%% Supplied to authors by publisher for camera-ready submission;
%% use defaults for review submission.
\acmJournal{PACMPL}
\acmVolume{1}
\acmNumber{POPL} % CONF = POPL or ICFP or OOPSLA
\acmArticle{1}
\acmYear{2020}
\acmMonth{1}
\acmDOI{} % \acmDOI{10.1145/nnnnnnn.nnnnnnn}
\startPage{1}

%% Copyright information
%% Supplied to authors (based on authors' rights management selection;
%% see authors.acm.org) by publisher for camera-ready submission;
%% use 'none' for review submission.
\setcopyright{none}
%\setcopyright{acmcopyright}
%\setcopyright{acmlicensed}
%\setcopyright{rightsretained}
%\copyrightyear{2018}           %% If different from \acmYear

%% Bibliography style
\bibliographystyle{ACM-Reference-Format}
%% Citation style
%% Note: author/year citations are required for papers published as an
%% issue of PACMPL.
\citestyle{acmauthoryear}   %% For author/year citations


%% Other packages
% \usepackage{amsmath}
% \usepackage{amssymb}
\usepackage{xspace}
\usepackage{listings}
\lstset{
  language=Caml,
  basicstyle=\ttfamily,
  keywordstyle=\ttfamily,
}
\lstnewenvironment{code}{
\lstset{
  language=Caml,
  basicstyle=\ttfamily,
  keywordstyle=\ttfamily,
}}
{}

\lstnewenvironment{ecode}{
\lstset{
  language=Caml,
  basicstyle=\ttfamily,
  keywordstyle=\ttfamily\bfseries,
  numbers=left,
  frame=leftline,
  xleftmargin=7.5mm,
  moredelim=[is][\bfseries]{==}{==},
  moredelim=[is][\underbar]{__}{__},
  moredelim=[is][\bfseries\underbar]{_=}{=_},
  escapeinside={(*@}{@*)},
}}
{}

\lstMakeShortInline{|}

%% Our commands
\usepackage{commands}

%% Some recommended packages.
\usepackage{booktabs}   %% For formal tables:
                        %% http://ctan.org/pkg/booktabs
\usepackage{subcaption} %% For complex figures with subfigures/subcaptions
                        %% http://ctan.org/pkg/subcaption


\begin{document}

%% Title information
\title[Short Title]{Full Title}         %% [Short Title] is optional;
                                        %% when present, will be used in
                                        %% header instead of Full Title.
% \titlenote{with title note}             %% \titlenote is optional;
%                                         %% can be repeated if necessary;
%                                         %% contents suppressed with 'anonymous'
% \subtitle{Subtitle}                     %% \subtitle is optional
% \subtitlenote{with subtitle note}       %% \subtitlenote is optional;
%                                         %% can be repeated if necessary;
%                                         %% contents suppressed with 'anonymous'


%% Author information
%% Contents and number of authors suppressed with 'anonymous'.
%% Each author should be introduced by \author, followed by
%% \authornote (optional), \orcid (optional), \affiliation, and
%% \email.
%% An author may have multiple affiliations and/or emails; repeat the
%% appropriate command.
%% Many elements are not rendered, but should be provided for metadata
%% extraction tools.

%% Author with single affiliation.
\author{First1 Last1}
\authornote{with author1 note}          %% \authornote is optional;
                                        %% can be repeated if necessary
\orcid{nnnn-nnnn-nnnn-nnnn}             %% \orcid is optional
\affiliation{
  \position{Position1}
  \department{Department1}              %% \department is recommended
  \institution{Institution1}            %% \institution is required
  \streetaddress{Street1 Address1}
  \city{City1}
  \state{State1}
  \postcode{Post-Code1}
  \country{Country1}                    %% \country is recommended
}
\email{first1.last1@inst1.edu}          %% \email is recommended

%% Author with two affiliations and emails.
\author{First2 Last2}
\authornote{with author2 note}          %% \authornote is optional;
                                        %% can be repeated if necessary
\orcid{nnnn-nnnn-nnnn-nnnn}             %% \orcid is optional
\affiliation{
  \position{Position2a}
  \department{Department2a}             %% \department is recommended
  \institution{Institution2a}           %% \institution is required
  \streetaddress{Street2a Address2a}
  \city{City2a}
  \state{State2a}
  \postcode{Post-Code2a}
  \country{Country2a}                   %% \country is recommended
}
\email{first2.last2@inst2a.com}         %% \email is recommended
\affiliation{
  \position{Position2b}
  \department{Department2b}             %% \department is recommended
  \institution{Institution2b}           %% \institution is required
  \streetaddress{Street3b Address2b}
  \city{City2b}
  \state{State2b}
  \postcode{Post-Code2b}
  \country{Country2b}                   %% \country is recommended
}
\email{first2.last2@inst2b.org}         %% \email is recommended


%% Abstract
%% Note: \begin{abstract}...\end{abstract} environment must come
%% before \maketitle command
\begin{abstract}
Text of abstract \ldots.

\end{abstract}
% "Each paper should have no more than 25 pages of text, excluding
% bibliography, using the new ACM Proceedings format."
% https://popl20.sigplan.org/track/POPL-2020-Research-Papers#POPL-2020-Call-for-Papers


%% 2012 ACM Computing Classification System (CSS) concepts
%% Generate at 'http://dl.acm.org/ccs/ccs.cfm'.
\begin{CCSXML}
<ccs2012>
<concept>
<concept_id>10011007.10011006.10011008</concept_id>
<concept_desc>Software and its engineering~General programming languages</concept_desc>
<concept_significance>500</concept_significance>
</concept>
<concept>
<concept_id>10003456.10003457.10003521.10003525</concept_id>
<concept_desc>Social and professional topics~History of programming languages</concept_desc>
<concept_significance>300</concept_significance>
</concept>
</ccs2012>
\end{CCSXML}

\ccsdesc[500]{Software and its engineering~General programming languages}
\ccsdesc[300]{Social and professional topics~History of programming languages}
%% End of generated code


%% Keywords
%% comma separated list
\keywords{keyword1, keyword2, keyword3}  %% \keywords are mandatory in final camera-ready submission


%% \maketitle
%% Note: \maketitle command must come after title commands, author
%% commands, abstract environment, Computing Classification System
%% environment and commands, and keywords command.
\maketitle


\section{Introduction}

1. ``We identify an important problem in the world.'' Be more specific than
just ``bugs''. Are we focusing on novices and students or are we focusing
on general software defects? Are we focusing on strongly-typed functional
languages or are we proposing something generic? What ``news article'' or
``survey paper'' citations can you list here to convince me that this is a
big deal?

2. ``Here are the properties that a good solution must have.'' Pick three.
Here are some examples: must be applicable to students; must produce
answers quickly; must produce answers that are very close to what humans
would do; must apply to programs from a wide range of application domains.

3. ``Here is the current state of the art. Note that each of these fails to
obtain at least one of the properties above.'' Candidates: manual debugging
(bad because of X and Y); using something like genprog (bad because of X
and Z); using pure fault localization (bad because of A and B); using delta
debugging or git/svn blame (bad because of P and Q); using something like
Nate or Sherrloc (bad because of Q and R).

4. ``Here are our two or three insights. These insights are the
underpinning of our solution.'' What are the most important ones?
Candidates: blame-labeled training sets are available; student repairs fall
into a reasonable number of categories (admitting a classification
technique); program repair can be viewed as a synthesis problem;
generalized ASTs can handle typed and untyped program manipulations.

5. ``We combine those insights into TECHNIQUE. It works by steps A, B and
C, which allow it to obtain the properties P1, P2, and P3 of a good
solution.'' Briefly condense the steps from Section 2 here.

6. ``We evaluate our technique. For Property P1, we use metric M1 and must
be at least as good as S1 to be successful. For Property P2, we use metric
M2 and must be at least as good as S2 to be successful. We obtain property
P3 by construction.'' Fill in the blanks. In addition, for every benchmark
set or human study used, indicate why you are sampling correctly --- why
those results are likely to generalize.

The contributions of this paper are as follows:
\begin{itemize}
  \item The algorithm. FIXME.
  \item The dataset. FIXME --- is this a contribution?
  \item The empirical evaluation. FIXME.
  \item The human study. FIXME.
\end{itemize}


\section{Overview}
\label{sec:overview}

We begin with an overview of our approach to suggesting fixes for various faulty
programs by collectively learning from the processes novice programmers follow
to fix errors in their programs.

\begin{figure}[ht]
\begin{ecode}
let rec mulByDigit i l =
  match l with
  | []     -> []
  | hd::tl -> (hd * i) @ mulByDigit i tl
\end{ecode}

\begin{ecode}
let rec mulByDigit i l =
  match l with
  | []     -> []
  | hd::tl -> [hd * i] @ mulByDigit i tl
\end{ecode}
\caption{(top) An ill-typed \ocaml program that should multiply each element
of a list by an integer. (bottom) The fixed version by the student.}
\label{fig:mulByDigit}
\end{figure}


\mypara{The Problem.} Consider the program \mbd shown at the top of
\autoref{fig:mulByDigit}, written by a student in an undergraduate Programming
course. The program is meant to multiply all the numbers in a list with an
integer digit. The student accidentally misuses the list append operator
(\texttt{@}), applying it to a number and a list rather than two lists.
%
Novice students who are still building a mental model of how the type checker
works are often perplexed by the compiler's error message \citep{Munson_2016}.
Hence a novice will often take a long time to arrive at a suitable fix, such as
the one shown at the bottom of \autoref{fig:mulByDigit}, where \texttt{@} is
used with a singleton list containing the multiplication of the head \texttt{hd}
and \texttt{i}.
%
Our goal is to use historical data of how programmers have fixed similar errors
in their programs to automatically and rapidly guide novices to come up with
candidate solutions like the one above.


\mypara{Solution: Analytic Program Repair.}
%
One approach is to view the search for candidate repairs
as a synthesis problem: synthesize a (small) set of edits
to the program that yields a good (\eg type-correct) one.
%
To ensure that synthesis is tractable, the search must be
carefully restricted to a relatively small, manually-constructed
``repair model'' that may not include the ``right'' fixes for
an erroneous program.
%FIXME: We are not manually constructing the repair model? I might make this
% sentence more clear
%
In this work, we present a novel strategy called
\emph{Analytic Program Repair} which breaks the
problem into three parts:
%
First, \emph{learn} a set of widely used \emph{fix templates}.
%
Second, \emph{predict}, for each erroneous program, the correct fix template to apply.
%
Third, \emph{synthesize} candidate repairs from the predicted template.

In the remainder of this section, we give a high-level overview
of our approach by describing how to:

\begin{enumerate}

  \item Represent fixes abstractly via \emph{fix templates}
        (\S~\ref{sec:overview:templates}),

  \item Acquire a \emph{training set} of labeled ill-typed programs and fixes
        (\S~\ref{sec:overview:data}),

  \item Learn a small set of candidate fix templates by \emph{partitioning}
        the training set (\S~\ref{sec:overview:learn}),

  \item Predict the appropriate template to apply by training a
        \emph{multi-class classifier} from the training set
        (\S~\ref{sec:overview:predict}), and

  \item Synthesize fixes by enumerating and checking terms from the
        predicted templates to give the programmer localized feedback
        (\S~\ref{sec:overview:synthesis}).
\end{enumerate}

\subsection{Representing Fixes}
\label{sec:overview:templates}

Our notion of a fix is defined as a \emph{replacement} of an existing expression
with a new \emph{candidate} expression at a specific program location. For
example, the \mbd program is fixed by replacing |(hd * i)| with the candidate
expression |[hd * i]| on line 4. We focus on AST-level replacements as they are
compact yet expressive enough to represent fixes.


\mypara{Generic Abstract Syntax Trees.}
%
We represent the different possible candidate expressions via abstract fix
templates called \emph{Generic Abstract Syntax Trees} (GAST) which each
corresponds to many possible candidate expressions.
%
GASTs are obtained from concrete ASTs in two steps.
%
First, we abstract concrete variable, function, and operator names.
%
Next, we prune GASTs at a certain depth $d$ to keep only the top-level changes
of the fix. Pruned sub-trees are replaced with \emph{holes}, which can represent
\emph{any} possible expression in our language.


Together, these steps ensure that GASTs only contain information about a fix's
\emph{structure} rather than the specific changes in variables and functions.
%
For example, the fix |[hd * i]| in the \mbd example would be represented by the
GAST of the expression |[_ $\oplus$ _]|, where variables |hd| and |i| are
abstracted into holes (\eg by pruning the GAST at a depth $d=2$) and |*| is
represented by an abstract binary operator. Our GAST approach is similar to
that of Lerner \emph{et al.}~\citep{Lerner2007-dt}, where AST-level modifications are used, but our proposed
GASTs will represent more abstract fix schemas.


\subsection{Acquiring a Fix-Labeled Training Set}
\label{sec:overview:data}

Previous work has used experts to create a set of ill-typed programs and their
fixed versions~\citep[][]{Lerner2007-dt, Loncaric2016-uk}, or to manually create
\emph{fix templates}~\cite{kim13par} that can yield \emph{repair
patches}~\citep[][]{martinez2013automatically,martinez2015mining}.
%
These approaches are hard to scale up to yield datasets suitable for machine
learning. Also, they do not discover the \emph{frequency} in practice of particular
classes of novice mistakes and their fixes.
%
In contrast, we show that such fix templates can be \emph{learned} from a large,
automatically constructed training set of ill-typed programs labeled with their
repairs.
%
Fixes in our dataset are represented as the ASTs of the expressions that students
changed in the ill-typed program to transform it into the correct solution.

\mypara{Interaction Traces.}
Following Seidel and Jhala~\citep{Seidel:2017}, we extract a labeled
dataset of erroneous programs
and their fixed versions from \emph{interaction traces}. Usually students write
several versions of their programs until they reach the correct solution for a
programming assignment. An instrumented compiler is used to capture such
sequences (or \emph{traces}) of student programs. The first type-correct
solution in this sequence of attempts is considered to be the fixed
version of all the previous ones and thus a pair for each of them is added to
the dataset. For each program pair, we then produce a \emph{diff} of their
abstract syntax trees (ASTs), and assign as the dataset's fix labels the
\emph{smallest} sub-tree that changed between the correct and ill-typed attempt
of the program.


\subsection{Learning Candidate Fix Templates}
\label{sec:overview:learn}

Each labeled program in our dataset contains a fix, which we abstract to a fix
template. For example, in the \mbd program from \autoref{fig:mulByDigit} we get
the fix candidate |[hd * i]| and hence the fix template |[_ $\oplus$ _]|.
However, a large dataset of fix-labeled programs, which may include many
diverse solutions, can introduce a huge set of fix templates, which can be
inappropriate for predicting the correct one to be used for the final program
repair.

Therefore, the next step in our approach is to learn a set of fix templates
that is \emph{small enough} to automatically predict which template to apply to
a given erroneous program, but nevertheless \emph{covers} most of the fixes that
arise in practice.

\mypara{Partitioning the Fixes.} We learn a suitable small set of fix
templates by \emph{partitioning} all the templates obtained from our dataset,
and then selecting a single GAST to represent the fix templates from each fix
template set.
%
The partitioning serves two purposes.
%
First, it identifies a small set of the most common fix templates which then
enables the use of discrete classification algorithms to predict which template
to apply to a new program.
%
Second, it allows for the principled removal of outliers that arise because
student submissions often contain non-standard or idiosyncratic solutions that
we do not wish to use for suggesting fixes.

Unlike previous repair approaches that have used clustering to group together
similar programs (e.g.,~\citep{Wang_2018, Gulwani_2018}), we partition our set of
fix templates into their \emph{equivalence classes} based on a fix similarity
relation.


\subsection{Predicting Templates via Multi-class Classification}
\label{sec:overview:predict}

Next, we train models that can correctly predict error locations and fix
templates for a given ill-typed program. We use these models to generate
candidate expressions as possible program fixes. To reduce the
complexity of predicting the correct fix templates and error locations, we
separate these problems and encode them into two distinct \emph{supervised
classification} problems.

\mypara{Supervised Multi-Class Classification.}
We propose using a \emph{supervised multi-class classification} problem for
predicting fix templates. A \emph{supervised} learning problem is one where,
given a labeled training set, the task is to learn a function that accurately
maps the inputs to output labels and generalizes to future inputs. In a
\emph{classification} problem, the function we are trying to learn maps inputs
to a discrete set of two or more output labels, called \emph{classes}.
Therefore, we encode the task of learning a function that will map
subexpressions of ill-typed programs to a small set of fix templates as a
\emph{multi-class} classification (MCC) problem.

\mypara{Feature Extraction.} The machine learning models that we will train
to solve our MCC problem expect datasets of labeled \emph{fixed-length vectors}
as inputs. Therefore, we define a transformation of fix-labeled programs to
fixed-length vectors. Similarly to Seidel \emph{et al.}~\citep{Seidel:2017}, we define a set of
feature extraction functions $f_1, \ldots, f_n$, that map program subexpressions
to a numeric value (or just $\{0, 1\}$ to encode a boolean property). Given a
set of feature extraction functions, we can represent a single program's AST as
a set of fixed-length vectors by decomposing the AST $e$ into a set of its
constituent subexpressions $\{e_1, \ldots, e_m\}$ and then representing each
$e_i$ with the $n$-dimensional vector $[f_1(e_i), \ldots, f_n(e_i)]$. This
method is known as a \emph{bag-of-abstracted-terms} (BOAT) representation in
previous work. % FIXME: CITE

\mypara{Predicting Templates via MCC.}
Our fix-labeled dataset can be updated so the labels represent the corresponding
template that fixes each location, drawn from the minimal set of fix templates
that were acquired through partitioning. We then train a \emph{Deep Neural
Network (DNN)} classifier on the updated template-labeled data set.

Neural networks have the advantage of associating each class with a
\emph{confidence score} that can be interpreted as the model's confidence of
each class being correct for a given input. This confidence score can be used to
rank fix-template predictions for new programs and use them in descending order
to synthesize repairs until our high-level goal is reached. %FIXME: I might respecify this goal
Exploiting recent advances in machine learning, we use deep and dense
architectures \citep{Schmidhuber_2015} for more accurate fix template predictions.

\mypara{Error Localization.} We view the problem of finding error locations
in a new program as a \emph{binary} classification problem. In contrast with the
template prediction problem, we want to learn a function that maps a program's
subexpressions to a binary output representing the presence of an error or not.
Therefore, this problem is a equivalent to a MCC with only two classes and thus,
we use similar deep architectures of neural networks. For each expression in a
given program, the learned model outputs a confidence score representing how
likely it is an error location that needs to be fixed. We exploit those scores
to synthesize candidate expressions for each location in descending order of
confidence.
% We use this approach, rather than standard fault localization, because FIXME.

\subsection{Synthesizing Feedback from Templates}
\label{sec:overview:synthesis}

Next, we utilize existing program synthesis techniques to \emph{synthesize}
candidate expressions that will be used to provide feedback to users.
Guided by our learned fix-template predictions and a set of
possible error locations to guide program synthesis, we return a ranked
list of \emph{minimal} repairs to users as feedback.

\mypara{Program Synthesis.} Given a set of locations and candidate templates for
those locations, we are trying to solve a problem of \emph{program synthesis}.
For each program location, we search over all possible
expressions in the language's grammar for a small set of candidate
expressions that match the fix template and make the program type-check.
Expressions from the ill-typed program are also used during synthesis
to prune the search space of candidate expressions.

\mypara{Synthesis for Multiple Locations.}
It is often the case that more than one location needs to be fixed. Therefore,
we do not only consider the ordered set of single error locations for synthesis,
but rather its power set. For simplicity, we consider fixing different program
locations as independent; the probability we assign that a set of locations
needs to be fixed is thus the product their individual confidence scores. This
is unlike recent approaches to multi-hunk program repair~\citep{Saha_2019}
where modifications depend on each other.

\mypara{Ranking Fixes.} Finally, we rank each solution by two metrics, the
\emph{tree-edit distance} and the \emph{string-edit} distance. Previous work
\citep{Lerner2007-dt, Wang_2018, Gulwani_2018} has used such metrics to consider
minimal changes, \ie changes that are as close as possible to the original
programs, so novice programmers are presented with more coherent feedback.
% Additionally, more experienced programmers might have in mind what \emph{type}
% they want their functions to be. We thus provide the user with the option to
% give the intended type for the program's functions we are trying to repair. In
% our evaluation, we acquire the intended types from the fixed versions of the
% dataset (see \autoref{sec:synthesis}).

\begin{figure}[ht]
  \begin{ecode}
  let rec mulByDigit i l =
    match l with
    | []     -> []
    | hd::tl -> [(*@$v_1$@*) * (*@$v_2$@*)] @ mulByDigit i tl
  \end{ecode}
  \caption{A candidate repair for the \mbd program.}
  \label{fig:repair}
  \end{figure}

\mypara{Example.} We see in \autoref{fig:repair} a minimal repair that our
method could return (|[$v_1$ * $v_2$]| in line 4), using the template discussed
in \S~\ref{sec:overview:learn} to synthesize this solution. While this solution
is not the highest-ranked that our implementation returns (which would be
identical to the human solution), it demonstrates relevant aspects of the
synthesizer. In particular, this solution has some abstracted variables, $v_1$
and $v_2$. Our algorithm suggests to the user that they can replace the two
variables with two distinct variables and insert the whole expression into a
list, in order to obtain the correct program. We hypothesize that such solutions
produced by our algorithm can provide valuable feedback to novices, and we
investigate that claim empirically in \S~\ref{sec:eval:useful}.

%%%
%%%  FIXME: This entire section is copy-pasted from NATE! Has to change a lot!
%%%

\section{Predicting Error Locations}
\label{sec:localization}

In this section, we briefly describe our approach to localizing type errors, by
introducing our feature set and the models we used. These we also be used for
predicting repair templates with some changes in \autoref{sec:templ-pred}.

Stemming from previous work, we define \lang (\autoref{fig:syntax}), a simple
lambda calculus with integers, booleans, pairs, and lists. For our error
localization algorithm, we aim to instantiate the $\blamesym$ function of
\autoref{fig:api}, which takes as input a $\Model$ of type errors and an
ill-typed program $e$, and returns an ordered list of subexpressions from $e$
paired with the confidence score $\Conf$ that they should be blamed.

A $\Model$ is produced by $\trainsym$, which performs supervised learning on a
training set of feature vectors $\V$ and (boolean) labels $\B$. Once trained, we
can $\evalsym$uate a $\Model$ on a new input, producing the confidence $\Conf$
that the blame label should be applied. We describe multiple $\Model$s and their
instantiations of $\trainsym$ and $\evalsym$ (\autoref{sec:models}).

Of course, the $\Model$ expects feature vectors $\V$ and blame labels $\B$, but
we are given program pairs. So our first step must be to define a suitable
translation from program pairs to feature vectors and labels, \ie we must define
the $\extractsym$ function in \autoref{fig:api}. We model features as
real-valued functions of terms, and extract a feature vector for each
\emph{subterm} of the ill-typed program (\autoref{sec:features}). Then we define
the blame labels for the training set to be the subexpressions that changed
between the ill-typed program and its subsequent fix, and model $\blamesym$ as a
function from a program pair to the set of expressions that changed
(\autoref{sec:labels}). The $\extractsym$ function, then, extracts $\featuresym$
from each subexpression and computes the blamed expressions according to
$\labelsym$.

\begin{figure}
\small
\centering
  \begin{minipage}[c]{\linewidth}
  \[
  \boxed{
  \begin{array}{rcl}
  e & ::=    & x \spmid \efun{x}{e} \spmid \eapp{e}{e} \spmid \elet{x}{e}{e} \\
    & \spmid & n \spmid \eplus{e}{e}\\
    & \spmid & b \spmid \eif{e}{e}{e} \\
    & \spmid & \epair{e}{e} \spmid \epcase{e}{x}{x}{e} \\
    & \spmid & \enil \spmid \econs{e}{e} \spmid \ecase{e}{e}{x}{x}{e} \\[0.05in]

  n & ::= &  0, 1, -1, \ldots \\[0.05in]

  b & ::= &  \etrue \spmid \efalse \\[0.05in]

  t & ::= & \alpha \spmid \tbool \spmid \tint \spmid \tfun{t}{t} \spmid \tprod{t}{t} \spmid \tlist{t} \\[0.05in]
  \end{array}
  }
  \]
  \captionof{figure}{Syntax of \lang}
  \label{fig:ml-syntax}
  \end{minipage}
  \begin{minipage}[c]{\linewidth}
    \[
    \boxed{
    \begin{array}{rcl}
    e & ::=    & x \spmid \_  \spmid \efun{x}{e} \spmid \eapp{e}{e} \spmid \elet{x}{e}{e} \\
      & \spmid & n \spmid \ebop{e}{e} \spmid \eif{e}{e}{e} \\
      & \spmid & \epair{e}{e} \spmid \epcase{e}{x}{x}{e} \\
      & \spmid & \enil \spmid \econs{e}{e} \spmid \ecase{e}{e}{x}{x}{e} \\[0.05in]
    \end{array}
    }
    \]
    \captionof{figure}{Syntax of \repairLang}
    \label{fig:rtl-syntax}
  \end{minipage}
\end{figure}



\subsection{Features}
\label{sec:features}
The first issue we must tackle is formulating our learning task in machine
learning terms. We are given programs over \lang, but learning algorithms expect
to work with \emph{feature vectors} $\V$ --- vectors of real numbers, where each
column describes a particular aspect of the input. Thus, our first task is to
convert programs to feature vectors.

We choose to model a program as a \emph{set} of feature vectors, where each
element corresponds an expression in the program. Thus, given the |sumList|
program in \autoref{fig:sumList} we would first split it into its constituent
sub-expressions and then transform each sub-expression into a single feature
vector. We group the features into five categories, using \autoref{tab:sumList}
as a running example of the feature extraction process.

\paragraph{Local syntactic features}
These features describe the syntactic category of each expression $e$. In other
words, for each production of $e$ in \autoref{fig:syntax} we introduce a feature
that is enabled (set to $1$) if the expression was built with that production,
and disabled (set to $0$) otherwise. For example, the \IsNil feature in
\autoref{tab:sumList} describes whether an expression is the empty list $\enil$.

We distinguish between matching on a list vs.\ on a pair, as this affects the
typing derivation. We also assume that all pattern matches are well-formed ---
\ie all patterns must match on the same type. Ill-formed match expressions would
lead to a type error; however, they are already effectively localized to the
match expression itself. We note that this is not a \emph{fundamental}
limitation, and one could easily add features that specify whether a match
\emph{contains} a particular pattern, and thus have a match expression that
enables multiple features.

\paragraph{Contextual syntactic features}
These are similar to local syntactic features, but lifted to describe the parent
and children of an expression. For example, the \IsCaseListP feature in
\autoref{tab:sumList} describes whether an expression's \emph{parent} matches on
a list. If a particular $e$ does not have children (\eg a variable $x$) or a
parent (\ie the root expression), we leave the corresponding features disabled.
This gives us a notion of the \emph{context} in which an expression occurs,
similar to the \emph{n-grams} used in linguistic models
\citep{Hindle2012-hf,Gabel2010-el}.

\paragraph{Expression size}
We also propose a feature representing the \emph{size} of each expression, \ie
how many sub-expressions does it contain? For example, the \ExprSize feature in
\autoref{tab:sumList} is set to three for the expression |sumList tl| as it
contains three expressions: the two variables and the application itself. This
allows the model to learn that, \eg, expressions closer to the leaves are more
likely to be blamed than expressions closer to the root.

\paragraph{Typing features}
A natural way of summarizing the context in which an expression occurs is with
\emph{types}. Of course, the programs we are given are \emph{untypeable}, but we
can still extract a \emph{partial} typing derivation from the type checker and
use it to provide more information to the model.

A difficulty that arises here is that, due to the parametric type constructors
$\tfun{\cdot}{\cdot}$, $\tprod{\cdot}{\cdot}$, and $\tlist{\cdot}$, there is an
\emph{infinite} set of possible types --- but we must have a \emph{finite} set
of features. Thus, we abstract the type of an expression to the set of type
constructors it \emph{mentions}, and add features for each type constructor that
describe whether a given type mentions the type constructor. For example, the
type $\tint$ would only enable the $\tint$ feature, while the type
$\tfun{\tint}{\tbool}$ would enable the $\tfun{\cdot}{\cdot}$, $\tint$, and
$\tbool$ features.

We add these features for parent and child expressions to summarize the context,
but also for the current expression, as the type of an expression is not always
clear \emph{syntactically}. For example, the expressions |tl| and |sumList tl|
in \autoref{tab:sumList} both enable \HasTypeList, as they are both inferred to
have a type that mentions $\tlist{\cdot}$.
%constructor.

Note that our use of typing features in an ill-typed program subjects us to
\emph{traversal bias} \citep{McAdam1998-ub}. For example, the |sumList tl|
expression might alternatively be assigned the type $\tint$. Our models will
have to learn good localizations in spite this bias (see
\autoref{sec:evaluation}).

\paragraph{Type error slice}
Finally, we wish to distinguish between changes that could fix the error, and
changes that \emph{cannot possibly} fix the error. Thus, we compute a minimal
type error \emph{slice} for the program (\ie the set of expressions that
contribute to the error), and add a feature that is enabled for expressions that
are part of the slice. The \InSlice feature in \autoref{tab:sumList} indicates
whether an expression is part of such a minimal slice, and is enabled for all of
the sampled expressions except for |tl|, which does not affect the type error.
If the program contains multiple type errors, we compute a minimal slice for
each error.

In practice, we have found that \InSlice is a particularly important feature,
and thus include a post-processing step that discards all expressions where it
is disabled. As a result, the |tl| expression would never actually be shown to
the classifier. We will demonstrate the importance of \InSlice empirically in
\autoref{sec:feature-utility}.


\subsection{Labels}
\label{sec:labels}
Recall that we make predictions in two stages. First, we use $\evalsym$ to
predict for each subexpression whether it should be blamed, and extract a
confidence score $\Conf$ from the $\Model$. Thus, we define the output of the
$\Model$ to be a boolean label, where ``false'' means the expression
\emph{should not} change and ``true'' means the expression \emph{should} change.
This allows us to predict whether any individual expression should change, but
we would actually like to predict the \emph{most likely} expressions to change.
Second, we \emph{rank} each subexpression by the confidence $\Conf$ that it
should be blamed, and return to the user the top $k$ most likely blame
assignments (in practice $k=3$).


We identify the fixes for each ill-typed program with an expression-level
diff~\citep{Lempsink2009-xf}. We consider two sources of changes. First, if an
expression has been removed wholesale, \eg if $\eapp{f}{x}$ is rewritten to
$\eapp{g}{x}$, we will mark the expression $f$ as changed, as it has been
replaced by $g$. Second, if a new expression has been inserted \emph{around} an
existing expression, \eg if $\eapp{f}{x}$ is rewritten to
$\eplus{\eapp{f}{x}}{1}$, we will mark the application expression $\eapp{f}{x}$
(but not $f$ or $x$) as changed, as the $+$ operator now occupies the original
location of the application.


\subsection{Learning Algorithm}
\label{sec:models}
\lstDeleteShortInline{|} % sigh...

Recall that we formulate type error detection at a single expression as a
supervised classification problem. This means that we are given a training data
set
%$S = \{ \langle v_i, b_i \rangle \}_{i=1}^n$
$S : \List{\V \times \B}$ of labeled examples, and our goal is to use it to
build a \emph{classifier}, \ie a rule that can predict a label $b$ for an input
$v$. Since we apply the classifier on each expression in the program to
determine those that are the most likely to be type errors, we also require the
classifier to output a \emph{confidence score} that measures how sure the
classifier is about its prediction.


There are many learning algorithms to choose from, existing on a spectrum that
balances expressiveness with ease of training (and of interpreting the learned
model). In this section we consider a standard learning algorithm: \emph{neural
networks}. A thorough introduction to these techniques can be found in
introductory machine learning textbooks \citep[\eg][]{Hastie2009-bn}.

Below we briefly introduce each technique by describing the rules it learns, and
summarize its advantages and disadvantages. For our application, we are
particularly interested in three properties -- expressiveness, interpretability
and ease of generalization. Expressiveness measures how complex prediction rules
are allowed to be, and interpretability measures how easy it is to explain the
cause of prediction to a human. Finally ease of generalization measures how
easily the rule generalizes to examples that are not in the training set; a rule
that is not easily generalizable might perform poorly on an unseen test set even
when its training performance is high.


\paragraph{Neural Networks}
The last (and most complex) model we use is a type of neural network called a
\emph{multi-layer perceptron} (see \citealt{Nielsen2015-pu} for an introduction
to neural networks). A multi-layer perceptron can be represented as a directed
acyclic graph whose nodes are arranged in layers that are fully connected by
weighted edges. The first layer corresponds to the input features, and the final
to the output.
%The output of a node $v$ in an internal layer is given by:
The output of an internal node $v$ is
\[ h_v = g\,(\sum_{j \in N(v)}\!W_{jv} h_j ) \] where $N(v)$ is the set of nodes
in the previous layer that are adjacent to $v$, $W_{jv}$ is the weight of the
$(j, v)$ edge and $h_j$ is the output of node $j$ in the previous layer. Finally
$g$ is a non-linear function, called the activation function, which in recent
work is commonly chosen to be the \emph{rectified linear unit} (ReLU), defined
as $g(x) = \mathsf{max}(0,x)$ \citep{Nair2010-xg}. The number of layers, the
number of neurons per layer, and the connections between layers constitute the
\emph{architecture} of a neural network. In this work, we use relatively simple
neural networks which have an input layer, a single hidden layer and an output
layer.

A major advantage of neural networks is their ability to discover interesting
combinations of features through non-linearity, which significantly reduces the
need for manual feature engineering, and allows high expressivity. On the other
hand, this makes the networks particularly difficult to interpret and also
difficult to generalize unless vast amounts of training data are available.

\lstMakeShortInline[mathescape=true]{|}


%% Acknowledgments
\begin{acks}                            %% acks environment is optional
                                        %% contents suppressed with 'anonymous'
  %% Commands \grantsponsor{<sponsorID>}{<name>}{<url>} and
  %% \grantnum[<url>]{<sponsorID>}{<number>} should be used to
  %% acknowledge financial support and will be used by metadata
  %% extraction tools.
  This material is based upon work supported by the
  \grantsponsor{GS100000001}{National Science
    Foundation}{http://dx.doi.org/10.13039/100000001} under Grant
  No.~\grantnum{GS100000001}{nnnnnnn} and Grant
  No.~\grantnum{GS100000001}{mmmmmmm}.  Any opinions, findings, and
  conclusions or recommendations expressed in this material are those
  of the author and do not necessarily reflect the views of the
  National Science Foundation.
\end{acks}


%% Bibliography
\bibliography{main}


%% Appendix
% \appendix
% \section{Appendix}

% Text of appendix \ldots

\end{document}
