\section{Overview}
\label{sec:overview}

\begin{figure}[ht]
    \begin{ecode}
      let rec mulByDigit i l =
        match l with
        | []     -> []
        | hd::tl -> (hd * i) @ (mulByDigit i tl)
    \end{ecode}

    \begin{ecode}
        let rec mulByDigit i l =
        match l with
        | []     -> []
        | hd::tl -> [hd * i] @ (mulByDigit i tl)
    \end{ecode}
    \caption{(top) An ill-typed \ocaml program that should multiply the elements of a list with a integer digit.
    (bottom) The fixed version by the student.}
    \label{fig:mulByDigit}
\end{figure}


Let’s start with an overview of our approach to suggesting fixes for various faulty programs by collectively learning
from the process novice programmers take to fix errors in their programs.

\mypara{The Problem.} Consider the \texttt{mulByDigit} program in \autoref{fig:mulByDigit} written by a student in an
undergraduate Programming Languages course. The program is meant to multiply all the numbers in a list with an integer
digit, but the student accidentally appends a number in the second case, rather than a list with that number. A
not-very-experienced programmer may be baffled by the compiler showing a type error there. They might confuse the
operator \texttt{@} (list append) that takes two lists as inputs with the \texttt{::}, which would be a solution as well
here. But the user decided to fix the error as shown in \autoref{fig:mulByDigit}. So, how do we propose the above
solution or something that would help the programmer arrive to such a solution?

\mypara{Solution. Suggesting Repairs via Supervised Classification and Program Synthesis.} Our approach is to view the
process of \emph{fixing} an erroneous program as a \emph{supervised multi-class classification problem}, whose results
are then fed to a \emph{program synthesizer}. A classification problem entails learning a function that maps inputs to a
discrete set of output labels (in contrast to regression, where the output is typically a real number). A supervised
learning problem is one where we are given a training set where the inputs and labels are known, and the task is to
learn a function that accurately maps the inputs to output labels and generalizes to new, yet-unseen inputs. To realize
the above approach for suggesting good possible repairs as a practical tool, we have to solve five sub-problems.

\begin{enumerate}
  \item How can we acquire a training set of blame-labeled ill-typed programs with the respective user-defined fixes?
  \item How can we represent possible solutions to an error?
  \item What are the appropriate models to train for our case?
  \item How can we use predictive models to repair faulty programs?
  \item How can we use predictive models and synthesized repairs to give localized feedback to the programmer?
\end{enumerate}


%%% FIXME: Taken mostly from Nate, needs many changes...
\subsection{Step 1: Acquiring a Blame-Labeled Training Set}
\label{subsec:step1}

The first step is to gather a training set of ill-typed programs, where each erroneous sub-term is explicitly labeled.
Prior work has often enlisted expert users to curate a set of ill-typed programs and then \emph{manually} determine the
correct fix~\citep[\eg][]{Lerner2007-dt, Loncaric2016-uk}. This method is suitable for evaluating the quality of a
localization (or repair) algorithm on a small number (e.g. 10s–100s) of programs. However, in general it requires a
great deal of effort for the expert to divine the original programmer’s intentions. Consequently, is difficult to scale
the expert-labeling to yield a dataset large enough (e.g. 1000s of programs) to facilitate machine learning. More
importantly, this approach fails to capture the frequency with which errors occur in practice.

\mypara{Interaction Traces.} We solve both the scale and frequency problems by instead extracting blame-labeled data
sets from \emph{interaction traces}. Software development is an iterative process. Programmers, perhaps after a lengthy
(and sometimes frustrating) back-and-forth with the type checker, eventually end up fixing their own programs. Previous
work has used an instrumented \ocaml compiler to record this conversation, i.e. record the sequence of programs
submitted by each programmer and whether or not it was deemed type-correct. For each ill-typed program in a particular
programmer’s trace, they find the first subsequent program in the trace that type checks and declare it to be the fixed
version. In our case, we used an existing extracted dataset~\citep[][]{yunounderstand, Seidel:2017} of ill-typed
programs and their fixes. From these pairs, we can extract a \emph{diff} of the abstract syntax trees (ASTs), and then
assign the blame labels to the \emph{smallest} sub-tree in the diff.

\mypara{Example.} Suppose our student fixed the \texttt{mulByDigit} program as shown above by adding a \texttt{[]}
around the result of the multiplication, the diff would include the list expression. Thus we would determine that the
list expression is the repair we have to blame.

\mypara{Bags-of-Abstracted-Terms.} Our representation of programs is parameterized by a set of feature abstraction
functions, (abbreviated to feature abstractions) $f_1, \ldots, f_n$ , that map terms to a numeric value (or just $\{0,
1\}$ to encode a boolean property). Given a set of feature abstractions, we can represent a single program’s AST as a
\emph{bag-of-abstracted-terms} (BOAT) by:
%
(1) decomposing the AST (term) $t$ into a bag of its constituent sub-trees (terms) $\{t_1, \ldots, t_m\}$; and then
%
(2) representing each sub-term $t_i$ with the $n$-dimensional vector $[f_1(t_i), \ldots, f_n(t_i)]$. Working with ASTs
is a natural choice as type-checkers operate on the same representation.

\mypara{Modeling Contexts.} Each expression occurs in some surrounding \emph{context}, and we would like the classifier
to be able make decisions based on the context as well. The context is particularly important for our task as each
expression imposes typing constraints on its neighbors. For example, a |@| operator tells the type checker that both
children must have type |'a list| and that the parent must accept an |'a list|. The BOAT representation makes it easy to
incorporate contexts: we simply \emph{concatenate} each term’s feature vector with the \emph{contextual features} of its
parent and children.

\mypara{Type features.} Another way to summarize the context in which an expression occurs is with types. Of course, the
programs we are given are untypeable, but we can still extract a partial typing derivation from the type checker and use
it to provide more information to the model. However, to help later our classifier give better predictions of the
possible fixes, we want those types to be as close as possible to the types that correct program would have or at least
be a super-type of them. To achieve that, we replace each time one location of the program with a typed hole and extend
the type checker to infer the type of that hole from the context of the program. This procedure would give more accurate
super-types than getting partial typing derivation from the untypeable original program.



\subsection{Step 2: Representing fixes as labels}
\label{subsec:step2}

Next, we must find a way to represent different fixes as a limited number of templates that would be easy for
programmers to understand and then proceed to fix their code. For that, we will use the abstract syntax trees of the
programs again.

\mypara{Clustering the Fixes.} Our dataset has now the erroneous programs and the fixes the users gave for the
respective programs. But those fixes can be arbitrary replacements of code, with different lengths, variable names,
functions etc. One thing is to do, of course is to remove outliers, i.e. programs that changed too much. But still, we
end up with a huge number of fixes all different from each other. Those, solutions must be put into a small number of
groups before they can be useful for any fix suggestion. Therefore, a simple clustering algorithm must be utilized that
will divide the fixes into groups of \emph{equivalent} fixes. But for that a notion of “similar” fixes must be defined.

\mypara{Generic Abstract Syntax Trees (GASTs).} The ASTs of the fixes is a good start, but those still contain too much
information. See our running example. |[hd * i]| is a list that includes a multiplication of the variables |hd| and |i|.
While we care about the list and the binary operator, all the rest information is not that important. Therefore, we
define \emph{GASTs}: essentially ASTs that keep only the basic structure of the trees, without keeping variable names,
operators etc. After, transforming each fix’s AST to a GAST, we prune those at a certain depth $d$ to keep only the
top-level changes of the code and replace possible children that were pruned away with \emph{holes}. Those holes
represent that \emph{any expression} is possible to be at that point. Using this representation, we have a better way to
compare all fixes and group them together.

\mypara{Templates.} After clustering fixes based on their pruned GASTs into a small number of groups, we can use those
to produce fix templates. Those templates would be incomplete pieces of code produced by the GASTs. For our example, the
GAST would be a list with a binary operator as a child, whose children in return would be holes, if we were to prune
GASTs at a depth $d = 2$. The binary operator would also be unknown at this point, thus creating a template of the form
|[_ # _]|. We see in \autoref{fig:suggestion} how that template would work to provide feedback to programmers, but we
discuss later how we can complete the template with program synthesis.

\begin{figure}[ht]
  \begin{ecode}
    let rec mulByDigit i l =
      match l with
      | []     -> []
      | hd::tl -> [_ # _] @ (mulByDigit i tl)
  \end{ecode}
  \caption{A possible template for the \texttt{mulByDigit} program.}
  \label{fig:suggestion}
\end{figure}



\subsection{Step 3: Training Predictive Models}
\label{subsec:step3}

Now we have a minimal set of fix templates that we want to predict from. We enumerate our templates, starting from 1,
and update the dataset to have as labels the number of the template that \emph{fixes} that location of the program. We
can now train a classifier that predicts fix templates given a ill-typed program. Because we still have multiple
templates (\emph{classes}) to choose from, we have a multi-class problem in our hands, as opposed to the more common
binary classification. We will treat error localization as binary classification problem.

\mypara{Error Localization.} Previous work has shown to give excellent accuracy on localizing type-errors using program
analytics techniques. In order to reduce the complexity of predicting the correct template and the location that it
should be applied from our multi-class classifier, we separate those two problems. We use a second DNN classifier to
associate locations with their probability to be fixed. Then, starting from the locations with the highest probabilities
to be fixed, we try to repair them using the template predictions for those particular locations.

\mypara{Multi-class Classification.} We choose models that can handle multiple classes as labels. Such models are Deep
Neural Networks (DNNs). But those models produce not only a predicted template-class, but they also associate a metric
that can be interpreted as the classifier’s confidence in its prediction. We use deep and dense architectures to give a
better confidence to each template, with more accuracy.



\subsection{Step 4: Program Repair}
\label{subsec:step4}

After making template predictions and error localization for ill-typed programs, we exploit existing program synthesis
techniques to fill holes and generic expressions that our templates have and return programs that type-check. This way,
we can also eliminate templates that wouldn’t actually work for a particular location.

\mypara{Program Synthesis.} Given a set of locations and candidate templates for those locations, we are trying to solve
a problem of synthesis, meaning that we try to generate code that would match the template’s GAST and make the program
type-check. For each location, we enumerate all possible expressions, until we find a small set that makes the program
to type-check. We try to use existing code in our synthesis algorithm, by considering subexpressions of the expressions
we try to replace.

\mypara{Synthesis for Multiple Locations.} Using our error localization predictions, we get a confidence for each
location in the type-error slice. Previous work has shown that just the top 3 locations from this set can solve up to
90\% of type-errors. But there are cases where more than one location needs to be fixed. Therefore, we rank the
confidence of the powerset of all locations. The confidence for a subset of locations can be acquired by the product of
each location’s confidence in the subset. This holds because we consider each location’s probability to be the correct
location to be fixed independent from other locations.



\subsection{Step 5: Generating Feedback}
\label{subsec:step5}

Finally, having generated automatic repairs for a given ill-typed program using our predictive models and program
synthesis, we want to use that to help users completely fix their programs and understand what the problem was in the
original program. To do so, we want to provide \emph{minimal} repairs to students, meaning repairs that are as close to
their original program but also catch the programmers intent for that piece of code. Since we have multiple fix
templates to choose from and many candidate locations to fix in a program, maybe the user would find it most useful to
get more than one suggestion, and those suggestions to be ranked according to some metric.

\mypara{Ranking Fixes.} We rank each solution by two metrics, the \emph{tree-edit distance} and the \emph{string-edit}
distance. Previous work has used those metrics to consider minimal changes, i.e. changes that are as close as possible
to the original programs, so novice programmers can better understand feedback. However, different metrics may give
betters fixes as we discuss later. Programmers usually have in mind what the \emph{type signature} the functions that
they write are supposed to have. So, it is reasonable to provide the option to the user to give the intended type for
the program's functions we are trying to repair. In our case, we acquire the intended types from the fixed versions of
the dataset.

\mypara{Example.} In \autoref{fig:repair} we can see a minimal repair that our method could return, using the template
and error location discussed in \autoref{subsec:step2} to synthesize this solution. However, this solution is not the
top one that are implementation would give (that would be identical to the solution the programmer gave), we use this
repair to demonstrate different aspects of the synthesizer. We can see that this solution is very close to the one that
the programmer finally came up with, but still has some holes. This time, however, there are some indications as to what
these holes should be. Here our synthesizer suggested two different variables should be used to fill those template
holes.

\begin{figure}[ht]
  \begin{ecode}
    let rec mulByDigit i l =
      match l with
      | []     -> []
      | hd::tl -> [_var1_ * _var2_] @ (mulByDigit i tl)
  \end{ecode}
  \caption{A candidate repair for the \texttt{mulByDigit} program.}
  \label{fig:repair}
\end{figure}
